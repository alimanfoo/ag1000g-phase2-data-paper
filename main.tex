\documentclass[a4paper,11pt,abstracton,hidelinks]{scrartcl}

\usepackage[margin=3cm]{geometry}
\usepackage{graphicx}
\usepackage[UKenglish]{babel}
\usepackage{csquotes}
\usepackage[style=numeric,citestyle=numeric,backend=biber,sorting=none,doi=false,url=false]{biblatex}
\usepackage{float}
\usepackage[export]{adjustbox}
\usepackage[T1]{fontenc}
\usepackage{lmodern}
\usepackage[textsize=tiny]{todonotes}
\usepackage[labelsep=period,font=small,labelfont=bf,format=plain]{caption}
\captionsetup[table]{
  position=above,
  belowskip=10pt,
  aboveskip=0pt,
}
\usepackage[group-separator={,}]{siunitx}
\usepackage{booktabs}
\usepackage{pdflscape}
\usepackage{tablefootnote}
\usepackage{authblk}
\usepackage{threeparttable}
\usepackage{afterpage}
\usepackage{lineno}
\linenumbers
\usepackage{setspace}
\usepackage{hyperref}
\doublespacing

\newcommand{\beginsupplement}{%
  \setcounter{table}{0}
  \renewcommand{\thetable}{S\arabic{table}}%
  \setcounter{figure}{0}
  \renewcommand{\thefigure}{S\arabic{figure}}%
}


% TODO create refs.bibmicrosoft 
\addbibresource{refs.bib}


\title{
Genome variation and population structure among 1,142 mosquitoes of the African malaria vector species \emph{Anopheles gambiae} and \emph{Anopheles coluzzii}
}



\author[1]{\small The \emph{Anopheles gambiae} 1000 Genomes Consortium}
\affil[1]{\footnotesize A list of consortium members appears at the end of the paper}

\begin{document}

\maketitle


%%%%%%%%%%%%%%%%%%%%%%%%%%%%%%%%%%%%%%%%%%%%%%%%%%%%%%%%%%%%%%%%%%%%%%%%%%%%%%%
%%%%%%%%%%%%%%%%%%%%%%%%%%%%%%%%%%%%%%%%%%%%%%%%%%%%%%%%%%%%%%%%%%%%%%%%%%%%%%%
\begin{abstract}


%%
Mosquito control remains a central pillar of efforts to reduce malaria burden in sub-Saharan Africa.
%
However, insecticide resistance is entrenched in malaria vector populations, and countries with high malaria burden face a daunting challenge to sustain malaria control with a limited set of surveillance and intervention tools. 
%
Here we report on the second phase of a project to build an open resource of high quality data on genome variation among natural populations of the major African malaria vector species \textit{Anopheles gambiae} and \textit{Anopheles coluzzii}. 
%
We analysed whole genomes of 1,142 individual mosquitoes sampled from the wild in 13 African countries, and a further 234 individuals comprising parents and progeny of 11 lab crosses. 
%
The data resource includes high confidence single nucleotide polymorphism (SNP) calls at 57 million variable sites, genome-wide copy number variation calls, and haplotypes phased at biallelic SNPs.
%
We used the SNP data to analyse genetic population structure, compute allele frequencies, and characterise genetic diversity within and between populations.
%
We illustrate the utility of these data by investigating species differences in isolation by distance, genetic variation within proposed gene drive target sequences, and patterns of resistance to pyrethroid insecticides.
%
This data resource provides a foundation for developing new operational systems for molecular surveillance, and for accelerating research and development of new vector control tools.
%%

\end{abstract}


%%%%%%%%%%%%%%%%%%%%%%%%%%%%%%%%%%%%%%%%%%%%%%%%%%%%%%%%%%%%%%%%%%%%%%%%%%%%%%%
%%%%%%%%%%%%%%%%%%%%%%%%%%%%%%%%%%%%%%%%%%%%%%%%%%%%%%%%%%%%%%%%%%%%%%%%%%%%%%%
\section*{Introduction}


%% Need for molecular surveillance. Need for open genomic data resources to underpin surveillance.
%
The 10 countries with the highest malaria burden in Africa account for 65\% of all malaria cases globally, and attempts to reduce that burden are facing significant challenges @@REF.
%
Not least among these, resistance to pyrethroid insecticides is widespread throughout African malaria mosquito populations, potentially compromising the efficacy of mosquito control interventions which remain at the core of global malaria strategy @@REF.
%
There is a broad consensus that further progress cannot be made if interventions are applied blindly, but must instead be targeted and adapted based on data from surveillance of malaria parasite and mosquito populations @@REF.
%
To emphasize this, surveillance is now recommended as a core intervention @@REF, and national surveillance programmes are seeking the capability to capture high resolution molecular data alongside epidemiological and entomological variables.
%
Genome sequencing technologies are widely considered to be a key element of future surveillance systems, and projects are beginning to lay the ground work to enable the scale up and deployment of sequencing technology for public health operations in malaria-endemic countries.
%
These molecular surveillance systems will not work in isolation, but will depend on high quality open genomic data resources, including baseline data on genetic variation from multiple mosquito species and geographical locations, against which comparisons can be made and inferences regarding new events can be drawn.
%%

%% Need to accelerate malaria vector research and development of new vector control tools. Role of open genomic data in accelerating R&D.
%
Better surveillance can increase the impact and longevity of available mosquito control tools, but it is also widely recognised that sustaining malaria control will require the development and deployment of new control tools.
%
This includes repurposing existing insecticides not previously used in public health @@REF, developing entirely new insecticide classes @@REF, and developing tools that don't rely on insecticides, such as attractive toxic sugar baits @@REF and genetic modification of mosquito populations @@REF.
%
Research and development of new mosquito control tools has been greatly facilitated by the availability of open genomic data resources, including high quality genome assemblies and annotations, and more recently by high quality resources on genetic variation among natural mosquito populations @@REFs. 
%
Further expansion of these open data resources to incorporate unsampled mosquito populations and new types of genetic variation can provide new insights into a range of biological and ecological processes, and help to accelerate scientific discovery from basic biology through to operational research.
%%


%% Introduction to Ag1000G. Summary of Ag1000G phase 1. Limitations of Ag1000G phase 1.
%
The Anopheles gambiae 1000 Genomes project (Ag1000G - \url{https://www.malariagen.net/projects/ag1000g}) was established in 2013 to build a large scale open data resource on natural genetic variation in malaria mosquito populations.
%
The Ag1000G project forms part of the Malaria Epidemiology Network (MalariaGEN - \url{https://www.malariagen.net}), a data-sharing community of researchers investigating how genetic variation in humans, malaria parasites and malaria vectors can inform biology, epidemiology and control of malaria.
%
The first phase of the Ag1000G project released data from whole genome Illumina deep sequencing of mosquitoes from 8 African countries, including SNP calls and phased haplotypes \cite{Ag1000gConsortium2017}.
%
Mosquitoes were sampled from a broad geographical range, spanning Guinea-Bissau in West Africa to Kenya in East Africa.
%
Both \textit{Anopheles gambiae} and \textit{Anopheles coluzzii} were sampled, two closely related sibling species within the \textit{Anopheles gambiae} species complex \cite{coetzee2013}.
%
Genetic diversity was found to be high in nearly all populations, but there were marked patterns of population structure, and clear differences between populations in the magnitude and architecture of genetic diversity, indicating complex and varied demographic histories.
%
However, both of these species have a large geographical range \cite{wiebe2017}, and many countries and ecological settings are not represented in the Ag1000G phase 1 resource.
%
Also, only SNPs were studied in Ag1000G phase 1, but other types of genetic variation are known to be important.
%
In particular, copy number variation (CNV) has long been suspected to play a key role in insecticide resistance \cite{schimke1978, devonshire1991, weetman2015}, but no previous attempts to call genome-wide CNVs have been made in these species.
%%


%% Introduction to Ag1000G Phase 2.
%
This paper describes the data resource produced by the second phase of the Ag1000G project.
%
Within this phase, sampling and sequencing was expanded to include additional wild-caught mosquitoes sampled from five new regions not sampled in phase 1.
%
This includes three new locations with \textit{Anopheles coluzzii}, providing greater power for genetic comparisons with \textit{Anopheles gambiae}, and two island populations, providing a useful reference point to compare against mainland populations.
%
Seven new lab crosses are also included, providing a substantial resource for studying genome variation and recombination within known pedigrees.
%
In this phase we studied both SNPs and CNVs, and rebuilt a haplotype reference panel using all wild-caught specimens.
%
Here we describe the data resource, and use it to re-evaluate major population divisions and characterise genetic diversity.
%
We also illustrate the broad utility of the data by providing some preliminary insights into patterns of pyrethroid resistance; analyse genetic diversity within a gene in the sex-determination pathway currently targeted for gene drive development; and compare geographical population structure between the two mosquito species to investigate evidence for differences in dispersal behaviour.
%%


%%%%%%%%%%%%%%%%%%%%%%%%%%%%%%%%%%%%%%%%%%%%%%%%%%%%%%%%%%%%%%%%%%%%%%%%%%%%%%%
%%%%%%%%%%%%%%%%%%%%%%%%%%%%%%%%%%%%%%%%%%%%%%%%%%%%%%%%%%%%%%%%%%%%%%%%%%%%%%%
\section*{Results}


\subsection*{Population sampling and sequencing}


%% Paragraph on population sampling and sequencing.
%
We performed whole genome sequencing of 377 individual wild-caught mosquitoes, including individuals collected from 3 countries (The Gambia, Cote d'Ivoire, Ghana) and two offshore islands (Bioko Island, Mayotte Island) not represented in the previous project phase. 
%
We also sequenced 152 individuals comprising parents and progeny from seven lab crosses, where parents were drawn from the Ghana, Kisumu, Pimperena, Mali and Akron colonies. 
%
We then combined these data with the sequencing data previously generated during phase 1 of the project, to create a total resource of data from 1,142 wild-caught mosquitoes (1,058 female, 84 male) from 13 countries (Figure 1; Table 1) and 234 mosquitoes from 11 lab crosses (Table S1). 
%
As in the previous project phase, all mosquitoes were sequenced individually on Illumina technology using 100 bp paired-end reads to a target depth of 30x, and only mosquitoes obtaining a mean depth above 15x were included in the final resource. 
%
All mosquitoes were typed using conventional species diagnostic assays \cite{scott1993, fanello2002, santolamazza2008} and found to be either \textit{An. gambiae} or \textit{An. coluzzii} or to have an apparent \textit{An. gambiae}/\textit{An. coluzzii} hybrid genotype. 
%%


\begin{figure}[H]
	\begin{center}
		\includegraphics*[width=5.8in]{artwork/collection_site_map.jpg}
	\end{center}
	\caption{Ag1000G Phase 2 sampling locations. Colour of circle denotes species collected at location and area represents sample size. Colours on the map represent ecosystem classes; dark green designates forest ecosystems; see figure 9 in \cite{sayre2013} for a compete colour legend.}
	\label{sample_map}
\end{figure}


%% Table 1 - Population sampling.
%
\afterpage{%
%\clearpage
% N.B., for some reason using \newgeometry causes page number to get dropped from the subsequent page, so disable for now - not needed if using \footnotesize.
\newgeometry{margin=2cm}
\begin{landscape}
\thispagestyle{empty}
\begin{table}[h]
  \footnotesize
  \centering
  \begin{threeparttable}

  \caption{
%
\textbf{Ag1000G Phase 2 sampling locations}.
}

  \label{table:sampling_locations}

  
\begin{tabular}{lllcccccc}
\toprule
\multicolumn{6}{c}{\textbf{Collection}} &
\multicolumn{3}{c}{\textbf{\emph{Anopheles} species}}\\
\cmidrule(r){1-6}
\cmidrule(r){7-9}
Country & 
Location & 
Site &
Year &
Latitude & 
Longitude & 
\emph{gambiae} & 
\emph{coluzzii} & 
Unknown\\
\midrule

Angola & Luanda &  & 2009 & -8.8210 & 13.2910 & 0 & 78 & 0 \\

Burkina Faso & Bana &  & 2012 & 11.2330 & -4.4720 & 20 & 40 & 0 \\

 & Pala &  & 2012 & 11.1500 & -4.2350 & 46 & 10 & 0 \\

 & Souroukoudinga &  & 2012 & 11.2350 & -4.5350 & 26 & 25 & 0 \\

Cameroon & Daiguene &  & 2009 & 4.7770 & 13.8440 & 96 & 0 & 0 \\

 & Gado Badzere &  & 2009 & 5.7470 & 14.4420 & 73 & 0 & 0 \\

 & Mayos &  & 2009 & 4.3410 & 13.5580 & 105 & 0 & 0 \\

 & Zembe Borongo &  & 2009 & 5.7470 & 14.4420 & 23 & 0 & 0 \\

Cote d'Ivoire & Tiassale &  & 2012 & 5.8984 & -4.8229 & 0 & 71 & 0 \\

Equatorial Guinea & Bioko &  & 2002 & 3.7000 & 8.7000 & 9 & 0 & 0 \\

France & Mayotte & Bouyouni & 2011 & -12.7378 & 45.1417 & 1 & 0 & 0 \\

 &  & Combani & 2011 & -12.7787 & 45.1429 & 5 & 0 & 0 \\

 &  & Karihani Lake & 2011 & -12.7965 & 45.1217 & 3 & 0 & 0 \\

 &  & Mont Benara & 2011 & -12.8570 & 45.1552 & 2 & 0 & 0 \\

 &  & Mtsamboro Forest Reserve & 2011 & -12.7027 & 45.0811 & 1 & 0 & 0 \\

 &  & Mtsanga Charifou & 2011 & -12.9907 & 45.1557 & 8 & 0 & 0 \\

 &  & Sada & 2011 & -12.8521 & 45.1039 & 4 & 0 & 0 \\

Gabon & Libreville &  & 2000 & 0.3840 & 9.4550 & 69 & 0 & 0 \\

Gambia, The & Njabakunda & Kerr Birom Kardo & 2011 & 13.5500 & -15.9000 & 0 & 0 & 19 \\

 &  & Kerr Sama Kuma & 2011 & 13.5500 & -15.9000 & 0 & 0 & 8 \\

 &  & Maria Samba Nyado & 2011 & 13.5500 & -15.9000 & 0 & 0 & 18 \\

 &  & Sare Illo Buya & 2011 & 13.5500 & -15.9000 & 0 & 0 & 20 \\

Ghana & Koforidua &  & 2012 & 6.0945 & -0.2609 & 0 & 1 & 0 \\

 & Madina &  & 2012 & 5.6685 & -0.2193 & 12 & 12 & 0 \\

 & Takoradi &  & 2012 & 4.9122 & -1.7740 & 0 & 20 & 0 \\

 & Twifo Praso &  & 2012 & 5.6086 & -1.5493 & 0 & 22 & 0 \\

Guinea & Koraboh &  & 2012 & 9.2500 & -9.9170 & 22 & 0 & 0 \\

 & Koundara &  & 2012 & 8.5000 & -9.4170 & 18 & 4 & 0 \\

Guinea-Bissau & Antula &  & 2010 & 11.8910 & -15.5820 & 0 & 0 & 58 \\

 & Safim &  & 2010 & 11.9569 & -15.6492 & 0 & 0 & 33 \\

Kenya & Kilifi & Junju & 2012 & -3.8620 & 39.7450 & 0 & 0 & 16 \\

 &  & Mbogolo & 2012 & -3.6350 & 39.8580 & 0 & 0 & 32 \\

Uganda & Tororo & Nagongera & 2012 & 0.7700 & 34.0260 & 112 & 0 & 0 \\

\bottomrule
\end{tabular}



  \end{threeparttable}

\end{table}
\end{landscape}
\restoregeometry
} % end afterpage
%% end Table 1


\subsection*{Genome variation}


%% Paragraph on nucleotide variation, including SNP discovery, genome accessibility and haplotype phasing.
%
Sequence reads from all individuals were aligned to the AgamP4 reference genome \cite{Holt2002, sharakhova2007update} and SNPs were discovered using methods described previously \cite{Ag1000gConsortium2017}.
%
In total, we discovered 57,837,885 SNPs passing all variant quality filters. 
%
Of these high quality SNPs, 24\% were found to be multiallelic (three or more alleles), and 11\% were newly discovered in this project phase.
%
We also analysed genome accessibility to identify all genomic positions where read alignments were of sufficient quality and consistency to support accurate discovery and genotyping of nucleotide variation.
%
Similar to the previous project phase, we found that 61\% (140 Mbp) of genome positions were accessible, including 91\% (18 Mbp) of the exome and 58\% (121 Mbp) of non-coding positions.
%
Overall we discovered an average of one variant allele every 1.9 bases of the accessible genome.
%
We then used high quality biallelic SNPs to construct a new haplotype reference panel including all 1,142 wild-caught individuals, via a combination of read-backed phasing and statistical phasing as described previously \cite{Ag1000gConsortium2017}. 
%%


%%
In this project phase we also performed a genome-wide CNV analysis, described in detail elsewhere \cite{lucas2019}.
%
In brief, for each individual mosquito, we called CNVs by fitting a hidden Markov model to windowed data on depth of sequence read coverage, then compared calls between individuals to identify shared CNVs.
%
The CNV callset comprises 31,335 distinct CNVs, of which 7,086 were found in more than one individual, and 1,557 were present at at least 5\% frequency in one or more populations.
%
CNVs spanned more than 68 Mbp in total and overlapped 7,190 genes.
%
CNVs were significantly enriched in gene families associated with metabolic resistance to insecticides, suggesting that CNVs are an important mode of adaptation to mosquito control interventions.
%% 


\subsection*{Species assignment}


%% Paragraph about species ancestry, reporting results of AIM analysis.
%
The conventional molecular assay for differentiating \textit{An. gambiae} from \textit{An. coluzzii} is based on a fixed genetic difference at a single locus on the X Chromosome \cite{santolamazza2008}.
%
In the first phase of Ag1000G, we compared the results of this assay with genotypes at 506 ancestry-informative SNPs distributed across all chromosome arms, and found that in some cases the conventional assay was not concordant with alleles at other genome locations.
%
In particular, all individuals from two sampling locations (Kenya, Guinea-Bissau) carried a mixture of \textit{An. gambiae} and \textit{An. coluzzii} alleles, creating uncertainty regarding the appropriate species assignment \cite{Ag1000gConsortium2017}. 
%
Applying the same analysis to the new samples in Ag1000G Phase 2, we found that mosquitoes from The Gambia also carried a mixture of alleles from both species, in similar proportions to mosquitoes from Guinea-Bissau (Figure S1).
%
In all other locations, alleles at ancestry-informative SNPs were concordant with conventional diagnostics, except on Chromosome arm 2L where there has been a known introgression event carrying an insecticide resistance allele from \textit{An. gambiae} into \textit{An. coluzzii} \cite{weill2000, diabate2004, clarkson2014, norris2015}.
%
We observed this introgression in \textit{An. coluzzii} from both Burkina Faso and Angola in the phase 1 cohort, and it was also present among \textit{An. coluzzii} from Cote d'Ivoire, Ghana and Guinea in the Phase 2 cohort, confirming that this introgression has spread widely throughout \textit{An. coluzzii}.
%%


\begin{figure}[H]
	\begin{center}
		\includegraphics*[width=6.3in]{artwork/main_pca.jpeg}
	\end{center}
	\caption{Principal component analysis of the 1142 wild-caught mosquitoes derived from nucleotide variation in the accessible region of chromosome 3L.}
	\label{pca}
\end{figure}


\subsection*{Population structure}


%% Paragraph with results on population structure.
%
We investigated genetic population structure within the cohort of wild-caught mosquitoes by performing principal components analysis (PCA) (Figure 2).
%
To complement PCA, we also computed two measures of genetic differentiation -- average $F_{ST}$ and rates of rare variant sharing -- between all pairs of 16 populations defined by country of origin and species, excluding \textit{An. coluzzii} from Guinea due to small sample size (Figure 1, Table 1).
%
From these analyses, the three major groupings of samples from multiple locations were evident: \textit{An. coluzzii} from West Africa (Burkina Faso, Ghana, Cote d'Ivoire, Guinea); \textit{An. gambiae} from West and Central Africa (Burkina Faso, Ghana, Guinea, Cameroon, Bioko); far West (Guinea-Bissau, The Gambia). 
%
Within each of these groupings, samples clustered together in the PCA, and differentiation between sampling locations was weak, consistent with relatively unrestricted gene flow between locations.
%
Each of the remaining PCA clusters comprised samples from a single location and species (\textit{An. coluzzii} from Angola; \textit{An. gambiae} from Uganda; \textit{An. gambiae} from Gabon, \textit{An. gambiae} from Mayotte; Kenya), and each of these populations was relatively strongly differentiated from all other populations, consistent with geographical factors limiting gene flow.
%
A comparison of the two \textit{An. gambiae} island populations is interesting because Mayotte was highly differentiated, but individuals from Bioko clustered closely with other West African \textit{An. gambiae}, suggesting that Bioko is not isolated from continental populations despite a physical separation of more than 30 km.
%%


\begin{figure}[H]
	\begin{center}
		\includegraphics*[width=6.3in]{artwork/structure_composite.pdf}
	\end{center}
	\caption{\textbf{(a)} Average allele frequency differentiation (F\raisebox{-.4ex}{\scriptsize ST}) between pairs of populations. The bottom left triangle shows average F\raisebox{-.4ex}{\scriptsize ST} values between each population pair. The top right triangle shows the Z score for each F\raisebox{-.4ex}{\scriptsize ST} value estimated via a block-jackknife procedure. \textbf{(b)} Allele sharing in doubleton (\textit{f2}) variants. For each population, we identified the set of doubletons with at least one allele originating from an individual in that population. We then computed the fraction of those doubletons shared with each other population including itself. The height of the coloured bars represent the probability of sharing a doubleton allele between or within populations. Heights are normalized row-wise for each population so that the sum of coloured bars in each row equals 1.
}
	\label{fstdbl}
\end{figure}


%% Paragraph on isolation by distance.
%
The additional locations sampled in this project phase allow more comparisons to be made between the two species, and there are many open questions regarding their behaviour, ecology and evolutionary history.
%
For example, it would be valuable to know whether there are any differences in dispersal behaviour between the two species \cite{dao2014, epopa2017, north2019}. 
%
Providing a comprehensive answer to this question is beyond the scope of this study, but we performed a preliminary analysis by estimating Wright's neighbourhood size for each species \cite{wright1946}.
%
This statistic is an approximation for the effective number of potential mates for an individual, and can be viewed as a measurement of how genetic differentiation between populations correlates with the geographical distance between them (isolation by distance).
%
We used Rousset's method for estimating neighbourhood size based on a regression of normalised $F_{ST}$ against the logarithm of geographical distance \cite{rousset1997genetic}.
%
We found that average neighbourhood sizes are significantly lower in \textit{An. coluzzii} (median=85) than in \textit{An. gambiae} (median=261) (Figure \ref{fig:ibd_fig}) (@@TODO P value or 95\% CIs), indicating stronger isolation by distance among \textit{An. coluzzii} populations and suggesting a lower rate and/or range of dispersal.
%
We caution, however, that we do not have representation of both species at all sampling locations, and further sampling is needed to confirm this result.
%%


\begin{figure}[H]
	\begin{center}
		\includegraphics*[width=6.3in]{artwork/west_africa_multipanel_edit_NOD.pdf}
	\end{center}
	\caption{Isolation by distance in west African \textit{Anopheles}. \textbf{(a)} Study region and pairwise $F_{st}$. \textbf{(b)} Regressions of average genome-wide $F_{st}$ against geographic distance, following Rousset \cite{rousset1997}. Neighbourhood size is estimated as the inverse slope of the regression line. \textbf{(c)} Difference in neighbourhood size estimates by species, with each point representing a 200kb region of the genome.}
	\label{fig:ibd_fig}
\end{figure}


\subsection*{Genetic diversity}


%% Introduction to genetic diversity metrics.
%
The populations represented in the Ag1000G Phase 2 cohort can serve as a reference point for comparisons with populations sampled by other studies at other times and locations. 
%
To facilitate population comparisons, we characterised genetic diversity within each of 16 populations in our cohort -- defined by country of origin and species -- by computing a variety of summary statistics using data from the whole genome.
%
These statistics included nucleotide diversity ($\theta_{\pi}$; Figure 5a), Watterson's theta ($\theta_{W}$; Figure S2), Tajima's $D$ (Figure 5b) and site frequency spectra (SFS; Figure S3). 
%
We also estimated runs of homozygosity (ROH; Figure 5c) within each individual and runs of identity by descent (IBD; Figure 5d) between individuals, both of which provide additional information about haplotype sharing and patterns of relatedness within populations. 
%%


\begin{figure}[H]
	\begin{center}
		\includegraphics*[width=6.3in]{artwork/diversity_composite.jpeg}
	\end{center}
	\caption{Nucleotide diversity (pi)...}
	\label{div}
\end{figure}


%% Summary of main patterns in the diversity data.
%
The two Eastern-most populations (Kenya, Mayotte \textit{An. gambiae}) were outliers in all statistics calculated, with lower diversity, a deficit of rare variants relative to neutral expectation, and a higher degree of haplotype sharing within and between individuals.
%
The Kenyan population was represented in Ag1000G Phase 1, and we previously described how the patterns of diversity in this population were consistent with a severe and recent population bottleneck \cite{Ag1000gConsortium2017}.
%
The similarities between Kenya and Mayotte suggest that the Mayotte population has also experienced a population bottleneck, which would be expected given that Mayotte is an offshore island and may have been colonised by \textit{An. gambiae} via a small number of founding events.
%
Although ROH and IBD were elevated in both populations, Mayotte individuals had a larger number of shorter tracts than Kenyan individuals, which may reflect differences in the timing and/or strength of the bottleneck.
%
In contrast, the \textit{An. gambiae} individuals from Bioko Island had very similar patterns of diversity to \textit{An. gambiae} populations from West and Central Africa, supporting other evidence suggesting that this population is not isolated from continental populations.
%
The additional \textit{An. coluzzii} populations (Ghana, Cote d'Ivoire) were similar to the previously sampled Burkina Faso \textit{An. coluzzii} population, and the newly sampled Gambian population with uncertain species status was similar to the previously sampled Guinea-Bissau population, consistent with evidence from PCA that these populations form groupings with shared demographic histories and ongoing gene flow.
%%


\subsection*{Design of Cas9 gene drives}


%%
%
One of the ways the nucleotide variation data in this resource is being used is to inform the development of gene drives, a novel mosquito control technology using engineered selfish genetic elements to cause mosquito population suppression or modification \cite{burt2003, gantz2015, hammond2016, eckhoff2017, kyrou2018}.
%
Promising results have been obtained with a Cas9 homing endonuclease gene drive targeting a locus in the doublesex gene (\textit{dsx}), which is a critical component of the sex determination pathway \cite{kyrou2018}.
%
This locus was chosen in part because it has extremely low genetic diversity both within and between species in the \textit{An. gambiae} complex \cite{Ag1000gConsortium2017}.
%
Low diversity is required because any natural variation within the target sequence could inhibit association with the Cas9 guide RNA and cause resistance to the gene drive \cite{unckless2017}.
%
We reviewed nucleotide variation within \textit{dsx} using the expanded cohort of wild-caught samples in the Phase 2 cohort.
%
We found no new nucleotide variants within the target sequence used by Kyrou \textit{et. al.}, other than the previously known SNP at 2R:48714641, which has been shown not to interfere with the gene drive process \cite{kyrou2018}.
%
To facilitate the search for other potential gene drive targets in \textit{dsx} and other genes, we computed allele frequencies for all SNPs in all populations and included those data in the resource (@@link to ftp?).
%
We also compiled a table of all potential Cas9 target sites (23 bp regions with a protospacer-adjacent motif) in the genome, ordered by nucleotide diversity (@@link to ftp?).
%
This table includes @@N viable Cas9 targets that overlap a gene coding sequence and that have complete nucleotide conservation within the Ag1000G Phase 2 cohort.
%%


%%
The presence of highly conserved sequences within \textit{dsx} is also relevant to the molecular biology of sex determination in \textit{An. gambiae}, which remains poorly understood.
%
Insect sex determination depends on sex-specific splicing of \textit{dsx}, with exon 5 being retained in the female transcript and skipped in the male transcript @@REF.
%
In \textit{Drosophila melanogaster} this differential splicing depends on female-specific regulatory factors which bind to the exon 5 region of the \textit{dsx} pre-mRNA and promote inclusion of exon 5 within the final transcript @@REFs.
%
Any sequences required for the proper binding of this splicing machinery to the pre-mRNA will be subject to strong purifying selection, because disruption of \textit{dsx} splicing is highly deleterious in females @@REF. 
%
\textit{D. melanogaster} female splice enhancing factors bind to a series of six 13 bp sequences (\textit{dsx}REs) located in the untranslated region of exon 5 @@REF.
%
Putative homologs of these \textit{dsx}REs are present in \textit{An. gambiae},  located towards the end of exon 5 near the boundary with intron 6 @@REF.
%
We examined diversity within the putative \textit{dsx}REs and found that five out of six are located in tracts of near-complete nucleotide conservation, consistent with purifying selection and an active role in pre-mRNA binding (Figure \ref{fig:dsx}b).
%
However, the region of conservation at the intron 4/exon 5 boundary targeted for gene drive by Kyrou \textit{et. al.} \cite{kyrou2018} is mysterious, because it is distant from any of these known regulatory sequences.
%
This region of conservation in fact extends over a 200 bp region, including ~50 bp of untranslated sequence within exon 5, the entire 1,267bp coding sequence of exon 5, and 50 bp of intron 4 (Figure \ref{fig:dsx}a).
%
The conservation of both coding and non-coding sites within this region, extending well beyond the immediate vicinity of the splice junction, suggests it is also somehow involved in sex-specific splicing of exon 5, and adds further evidence for key differences in the molecular biology of sex determination between \textit{Anopheles} and \textit{Drosophila} @@REF. 
%


\subsection*{Resistance to pyrethroid insecticides}


%%
Malaria control in Africa depends heavily on mass distribution of long-lasting insecticidal bed-nets (LLINs) impregnated with pyrethroid insecticides \cite{Ranson2016, churcher2016}.
%
Entomological surveillance programs regularly test malaria vector populations for pyrethroid resistance using standardised bioassays, and these data have shown that pyrethroid resistance has become widespread in \textit{An. gambiae} \cite{Hemingway2016}.
%
However, pyrethroid resistance can be conferred by different molecular mechanisms, and it is not known which molecular mechanisms are responsible for resistance in which populations.
%
The nucleotide variation data in this resource include @@N non-synonymous SNPs within the \textit{Vgsc} gene that encodes the binding target for pyrethroid insecticides, of which two SNPs (\texttt{L995F}, \texttt{L995S}) are known to confer a pyrethroid resistance phenotype, and one SNP (\texttt{N1570Y}) has been shown to substantially increase pyrethroid resistance when present in combination with \texttt{L995F} \cite{jones2012}. 
%
These SNPs can serve as markers of target-site resistance to pyrethroids, but no study has previously identified any genetic markers of metabolic resistance in \textit{An. gambiae}.
%
Metabolic resistance to pyrethroids is mediated at least in part by increased expression of mixed-function oxidase (MFO) enzymes @@REFs, and we found large numbers of independent CNV alleles at these loci \cite{lucas2019}.
%
One MFO locus is on chromosome arm 2R and overlaps a cluster of 10 genes, including \textit{Cyp6p3} previously shown to metabolise pyrethroids \cite{muller2008}.
%
The second MFO locus occurs on the X chromosome and spans a single gene, \textit{Cyp9k1} which has also previously be shown to  metabolise pyrethroids \cite{vontas2018}
%
At these loci we found genes which overlapped with at least 15 distinct CNV alleles, several of which had reached population frequencies above 50\% and were associated with signatures of positive selection \cite{lucas2019}.
%
The phenotype of these CNVs remains to be confirmed, but given the multiple lines of evidence it seems reasonable to assume that CNVs at either of these loci can serve as a molecular marker of metabolic resistance to pyrethroids.
%%


%%
\begin{figure}[H]
	\begin{center}
		\includegraphics*[width=6.3in]{artwork/pyrethroid_resistance_simplified.jpg}
	\end{center}
	\caption{Pyrethroid resistance genotypes. The geographical distribution of four pyrethroid insecticide resistance genotype states for all 1142 Phase 2 wild caught mosquitoes are shown by country across Sub-Saharan Africa. The four pie chart colours represent genotype state: purple - these individuals were either homozygous or heterozygous for at least one of the two widespread \textit{kdr} pyrethroid target site mutations \textit{Vgsc-995F/S}; yellow - these individuals carried any genome amplification within either the \textit{Cyp6} or \textit{Cyp9} gene clusters; orange - these individuals carried at least one \textit{kdr} mutation and one \textit{Cyp} cluster amplification; grey - these individuals carried no known pyrethroid resistance genotypes (no \textit{kdr} or \textit{Cyp} amplifications).}
	\label{ir}
\end{figure}
%%


%%
We constructed an overview of the prevalence of these two pyrethroid resistance mechanisms within the Ag1000G Phase 2 cohort by combining the data on nucleotide and copy number variation.
%
The sampling of these populations was conducted at different times in different locations, and the geographical sampling is relatively sparse, so we cannot draw any general conclusions about the current distribution of resistance from our data.
%
However, some patterns were clear.
%
For example, West African populations of both species (Burkina Faso, Guinea, Cote d'Ivoire) all had more than 84\% of individuals carrying both target-site and metabolic resistance markers.
%
Towards Central Africa (Ghana, Cameroon, Gabon, Angola), target-site resistance was nearly fixed in all populations, but metabolic resistance markers were at lower frequencies, and the samples from Bioko Island carried no resistance markers at all.
%
The Bioko samples were collected in 2000, and so the lack of resistance may be related to the fact that sampling predated any major scale-up of vector control interventions.
%
However, the Gabon samples were also collected in 2000, and show that high levels of target-site resistance were present in at least some populations at that time.
%
In the Far West (Guinea Bissau, The Gambia), target-site resistance was absent, but metabolic resistance was present, and thus any molecular surveillance that assays only target site resistance at those locations could be missing an important signal.
%
In East Africa, both Kenya and Uganda had high frequencies of target-site resistance, but 81\% of Uganda individuals also had metabolic resistance, whereas only 2\% of Kenyans (one individual) carried both target site and metabolic resistance markers.
%
We know from previous analyses of Ag1000G data that target-site resistance has spread between these two populations \cite{Ag1000gConsortium2017}, and so it is interesting that metabolic resistance appears not to have spread so freely.
%
Denser spatio-temporal sampling and sequencing would enable us to build a more complete picture of the prevalence and spread of these different resistance mechanisms, and would be highly relevant to the design of insecticide resistance management plans.
%%



%%%%%%%%%%%%%%%%%%%%%%%%%%%%%%%%%%%%%%%%%%%%%%%%%%%%%%%%%%%%%%%%%%%%%%%%%%%%%%%
%%%%%%%%%%%%%%%%%%%%%%%%%%%%%%%%%%%%%%%%%%%%%%%%%%%%%%%%%%%%%%%%%%%%%%%%%%%%%%%
\section*{Discussion}


%%
In this paper we give an overview of Ag1000G Phase 2, a data release consisting of deep-sequenced whole genome variation from 1142 wild caught \emph{Anopheles} mosquitoes.
%
Genomic variation and meta-data are now available to download via FTP from www.malariagen.net for 16 distinct populations of mosquitoes, sampled from 13 countries across Sub-Saharan Africa. 
%
We defined 61\% (140Mbp) of the genome to be accessible (where we confidently call SNPs) and from these accessible regions we identified 57,837,885 high-quality SNPs, an average of one variant allele every 1.9 bases of accessible genome.
%
Ag1000G Phase 2 also sees the release of data from the parents and offspring of 11 colony crosses, we used the Mendelian errors detectable from parent offspring trios in the quality control process for wild caught samples but these also provide a valuable resource in calibration and validation of population genetic analyses more generally.
%%


%%
For ease of access and research, these data have been released in various analysis-ready formats.
%
Genotype calls for wild caught and lab colony samples are released in three data storage formats, VCF, HDF5 and Zarr, widely accepted as input types for population genomic software.
%
Due to the high diversity of \textit{Anopheles gambiae} and \textit{coluzzii}, raw genotype call files may be prohibitively large in some situations, for this reason we have also released these data in three versions (each in the three data formats): 
%
"raw" data containing all genotype calls, "pass" data containing calls that pass our hard filtering regime from accessible regions of the genome and "biallelic" data, potentially the most useful as both the smallest and the standard input for most population genetic analyses.
%
Biallelic data are the subset of pass data where variants have two alleles, a reference allele and a single alternative allele.
%
In Phase 2 this subset consists of 44,076,901 SNPs.
%
Whole genome phasing was performed using these biallelic variants, estimating chromosomal haplotypes for each sample, a data resource useful for many population genetic analyses including scans for selection.
%
Alongside meta-data for each sample containing details on where and when the sample was collected we have also added presence/absence information for known insecticide resistance loci.
%
These meta-data will therefore allow quick and easy access to population resistance frequencies without requiring bioinformatic tools. 
%%


%%
As evidenced by the research made possible through the first phase of Ag1000G \cite{Ag1000gConsortium2017} (\textit{e.g.} selection scans \cite{xue2019} and demographic modelling \cite{khatri2018}), this newly expanded genomic resource holds a huge potential for furthering our understanding of mosquito evolution and improving vector control.
%
We describe this data release using population genetic measures of diversity and structure and also demonstrate that, when applied genome-wide, these measurements can reveal medically relevant characteristics about insecticide resistance (PBO net distribution), vector control tool development (gene drive resistance) and fundamental vector biology (migration).
%
With the reliance on pyrethroid insecticides for ITNs and the ubiquity of pyrethroid resistance \cite{Hemingway2016}, it is essential for NMCPs to be able to access information about which regions might benefit from new (and more expensive \cite{churcher2016}) PBO bed-nets.
% 
By combining genomic data on loci associated with pyrethroid target-site and metabolic resistance, we were able to create a snapshot of putative resistance topology across Africa.
%
Results revealed that resistance topology fell broadly into three types (Figure \ref{ir}), each informative to the potential impact of PBO nets on vector populations:
%to vector control
\textbf{1.} Regions with only metabolic pyrethroid resistance alleles, where the PBO nets may drastically reduce the resistance phenotype.
%
\textbf{2.} Regions with both high target-site and high metabolic pyrethroid resistance associated loci frequencies, in these regions many individuals carry markers for both modes of resistance.
%to vector control
The outcome of PBO nets is harder to predict in these regions, as the addition of the synergist can only reduce the metabolic resistance mediated fraction of the pyrethroid resistance phenotype.
%
\textbf{3.} Regions with low frequency or no markers for metabolic resistance, here the combination of low potential benefits and the increased cost of the nets may render distribution inadvisable.
%%

%%
It must be reiterated that these results, though informative, are a prediction of pyrethroid resistance at the time of sample collection (Table \ref{table:sampling_locations}).
%
It is also likely that the four loci we analysed do not explain all phenotypic variation in all locations \cite{donnelly2016, mitchell2014} and, although all loci involve resistance associated genomic regions, the specific P450 CNVs used here have yet to undergo molecular validation \cite{lucas2019}.
%
In order to extract the maximum value from future genomic insecticide resistance monitoring we suggest a two stage approach.
%
Firstly, sampling should be conducted on a wider geographic scale, to improve understanding of population structure and diversity across sub-Saharan Africa, ensuring that genetically representative sentinel collection sites can be chosen and established.
%
Baseline frequencies for known resistance associated loci can then be established and putative resistance loci validated.
%
Secondly, samples should be collected regularly from sentinel sites across Africa, changes in frequencies of resistance associated loci can then be detected and this information can be quickly relayed to stakeholders to inform vector control programs.
%
This temporal genomic sampling also allows detection of new loci potentially linked to resistance by monitoring changes in frequency of alleles across the whole genome.
%%


%%
The expanded geographical sampling of Phase 2 allowed fine-scale analysis of how geographic distance between populations is related to genetic distance (isolation by distance), enabling inference of mosquito migration.
%
These analyses found much lower isolation by distance in \textit{An. gambiae} than \textit{An. coluzzii}, suggesting long distance gene flow, and therefore long distance migration in the former but not the latter species (concordant with conclusions made from the results of a study using time-series analyses \cite{dao2014}).
%
The higher genetic connectivity found in \textit{An. gambiae}, compared with \textit{coluzzii}, has important connotations for CRISPR Cas9 gene-drive mediated vector control as high migration in \textit{gambiae} may make these populations more susceptible to introgression of gene-drive resistance loci.
%
Gene-drive resistance can be caused by genetic diversity in the Cas9 binding target, natural variation here could inhibit the drive mechanism and thus constitute a resistant genotype that could spread and resist control (@@best ref?).
%
Utilising the Ag1000G Phase 1 data set, a gene-drive construct targeting the \textit{dsx} gene has been proposed for release into natural populations \cite{kyrou2018}.
%
Here, using the Phase 2 resources, we found no evidence of any novel naturally-occurring variation that could disrupt Cas9 binding at the \textit{dsx} target.
%%

%%
MalariaGen is committed to sequencing and releasing data that will make this approach to genomic insecticide resistance surveillance possible. 
%
Phase 3 of Ag1000G will see genome sequencing and related meta data released for @@n natural populations of \textit{Anopheles gambiae}, \textit{coluzzii} and \textit{arabiensis}, from @@n countries, filling gaps in current coverage including populations from central Africa.
%
The successor to Ag1000G, the Vector Observatory plans on sequencing 40,000 wild caught mosquitoes collected at sentinel sites across the world in both temporal and spatial transects.
%
These data will be curated and released in the same format as Ag1000G, generating unparalleled resources for vector management tools including genomic surveillance. 
%%


%%
Here, in this brief overview of Ag1000g Phase 2, we provide examples of how this rich resource of deep genome sequencing data can quickly, easily and relatively cheaply edify gene-drive vector control development, PBO net distribution and insecticide choice.
%
Our findings emphasize the value of large publicly available genomic data that can inform many aspects of malaria control while filling gaps in our understanding of mosquito evolution and elucidating basic biological/ecological processes.
%
Expanding and improving these repositories with higher sampling coverage and faster data turnaround can only strengthen the vector control armamentarium in the fight against malaria.
%%

%%%%%%%%%%%%%%%%%%%%%%%%%%%%%%%%%%%%%%%%%%%%%%%%%%%%%%%%%%%%%%%%%%%%%%%%%%%%%%%
%%%%%%%%%%%%%%%%%%%%%%%%%%%%%%%%%%%%%%%%%%%%%%%%%%%%%%%%%%%%%%%%%%%%%%%%%%%%%%%
\section*{Methods}


%%%%%%%%%%%%%%%%%%%%%%%%%%%%%%%%%%%%%%%%%%%%%%%%%%%%%%%%%%%%%%%%%%%%%%%%%%%%%%%
\subsection*{1. Population sampling}

%
Ag1000G Phase 2 mosquitoes were collected from natural populations at 33 sites from 23
geographical locations in 13 sub-Saharan African countries. 
%
Samples have been grouped into 16 populations based on collection country and species (Figure \ref{sample_map} \& Table \ref{table:sampling_locations}).
%
Throughout, we use species nomenclature following Coetzee \textit{et al}. \cite{coetzee2013};	
%
prior to	 Coetzee	 \textit{et al}., \textit{An. gambiae} was known as \textit{An. gambiae sensu stricto} (S form) and \textit{An. coluzzii} was known as \textit{An. gambiae sensu stricto} (M form).
%
Details of the eighteen collection sites novel to Ag1000G Phase 2 (dates, collection and DNA extraction methods) can be found in the Methods section.
%
Information pertaining to the collection of samples released as part of Ag1000G Phase 1 can be found in the supplementary methods of Ag1000G Consortium (2017) \cite{Ag1000gConsortium2017}.
%
Unless otherwise stated, the DNA extraction method used for the collections described below was Qiagen DNeasy Blood and Tissue Kit (Qiagen Science, MD, USA).
%%

%%%%%%%%%%%%%%%%%%%%%%%%%%%%%%%%%%%%%%%%%%%%%%%%%%%%%%%%%%%%%%%%%%%%%%%%%%%%%%%
\subsection*{1.1 Novel Phase 2 natural populations}
%
\textbf{C\^{o}te d'Ivoire (CIcol)}: Tiassal\'{e} (-4.823, 5.898) is located in the evergreen forest zone of southern C\^{o}te d'Ivoire.
%
The primary agricultural activity is rice cultivation in irrigated fields.
%
High malaria transmission occurs during the rainy seasons, between May and November.
%
Samples were collected as larvae from irrigated rice fields by dipping between May and Septermber 2012.
%
All larvae were reared to adults and females preserved over silica for DNA extraction.
%
Specimens from this site were all \textit{An. coluzzii}, determined by PCR assay \cite{santolamazza2008}
%%

\textbf{Bioko Island - Equatorial Guinea (GQgam)}: Collections were performed during the rainy season in September, 2002 by overnight CDC light traps in Sacriba of Bioko island (8.7, 3.7).
%
Specimens were stored dry on silica gel before DNA extraction.
%
Specimens contributed from this site were \textit{An. gambiae} females, genotype determined by two assays \cite{scott1993, santolamazza2004}.
%
All specimens had the 2L\textsuperscript{+a}/2L\textsuperscript{+a} karyotype as determined by the molecular PCR diagnostics \cite{White2007}. 
%
These mosquitoes represent a population that inhabited Bioko Island before a comprehensive malaria control intervention initiated in February 2004 \cite{Sharp2007}. 
%
After the intervention \textit{An. gambiae} was declining, and more recently almost only \textit{An. coluzzii} can be found \cite{Overgaard2012}.
%%

%
\textbf{Mayotte Island - France (FRgam)}: Samples were collected as larvae during March-April 2011 in temporary pools by dipping, in Bouyouni (-12.738, 45.143), M'Tsamboro Forest Reserve (-12.703, 45.081), Combani (-12.779, 45.143), Mtsanga Charifou (-12.991, 45.156), Karihani Lake forest reserve (-12.797, 45.122), and Sada (-12.852, 45.104) in Mayotte island.
%
Larvae were stored in 80\% ethanol prior to DNA extraction. 
%
All specimens contributed to Ag1000G Phase 2 were \textit{An. gambiae} \cite{santolamazza2004} with the standard 2L\textsuperscript{+a}/2L\textsuperscript{+a} or inverted 2L\textsuperscript{a}/2L\textsuperscript{a} karyotype as determined by the molecular PCR diagnostics \cite{White2007}.
%
The samples were identified as males or females by the sequencing read coverage of the X chromosome using LookSeq \cite{Manske2009}.
%%

%
\textbf{The Gambia (GM)}: Indoor resting female mosquitoes were collected by pyrethrum spray catch from four hamlets around Njabakunda (-15.90, 13.55), North Bank Region, The Gambia between August and October 2011.
%
The four hamlets were Maria Samba Nyado, Sare Illo Buya, Kerr Birom Kardo, and Kerr Sama Kuma; all are within 1 km of each other.
%
This is an area of unusually high hybridization rates between \textit{An. gambiae s.s.} and \textit{An. coluzzii} \cite{Caputo2008, Nwakanma2013}.
%
Njabakunda village is approximately 30 km to the west of Farafenni town and 4 km away from the Gambia River.
%
The vegetation is a mix of open savannah woodland and farmland.
%
With apparent high gene-flow in the region, it is problematic to assign species to these samples.
%%

%
\textbf{Ghana (GHcol/GHgam)}: Twifo Praso (5.609, -1.549) a peri-urban community located in semi-deciduous forest in the Central Region of Ghana.
%
It is an extensive agricultural area characterised by small-scale (vegetable growing) and large-scale commercial farms such as oil palm and cocoa plantations.
%
Mosquito samples were collected as larvae from puddles near farms between September and October, 2012.
%
Madina (5.668,	-0.219) is suburb of Accra within a coastal savanna zone of Ghana. 
%
It is an urban community characterised by myriad vegetable-growing areas.
%
The vegetation consists of mainly grassland interspersed with dense short thickets often less than 5 m high with a few trees.
%
Specimens were sampled from puddles near roadsides and farms between October and December 2012.
%
Takoradi (4.912, -1.774) is the capital city of Western Region of Ghana.
%
It is an urban community located in the coastal savanna zone.
%
Mosquito samples were collected from puddles near road construction and farms between August and September 2012.
%
Koforidua (6.094, -0.261) is a capital city of Eastern Region of Ghana and is located in semi-deciduous forest. 
%
It is an urban community characterized by numerous small-scale vegetable farms. 
%
Samples were collected from puddles near road construction and farms between August and September 2012.
%
Larvae from all collection sites were reared to adults and females preserved over silica for DNA extraction.
%
Both \textit{An. gambiae} and \textit{An. coluzzii} were collected from these sites, determined by PCR assay \cite{santolamazza2008}.
%%

%
\textbf{Guinea-Bissau (GW)}: Mosquitoes were collected in October 2010 using indoor CDC light traps, in the village of Safim (11.957, -15.649), ca. 11 km north of Bissau city, the capital of the country.
%
Malaria is hyperendemic in the region and transmitted by members of the Anopheles gambiae complex (Vicente et al., 2017).
%
\textit{Anopheles arabiensis, An. melas, An. coluzzii} and \textit{An. gambiae}, as well as hybrids between the latter two species, are known to occur in the region \cite{Gordicho2014, Vicente2017}.
%
Mosquitoes were preserved individually on 0.5ml micro-tubes filled with silica gel and cotton. DNA extraction was performed by a phenol-chloroform protocol \cite{Donnelly1999}.
%
Guinea-Bissau is another region where defining species is problematic (Vicente), so no species has been assigned here.
%%

%%%%%%%%%%%%%%%%%%%%%%%%%%%%%%%%%%%%%%%%%%%%%%%%%%%%%%%%%%%%%%%%%%%%%%%%%%%%%%%
\subsection*{1.2 Colony populations}

%%
The Ag1000G Phase 2 data release includes the genomes of seven additional lab colony crosses, both parents and offspring (Table \ref{table:colony crosses}):
%
cross 18-5 (Ghana mother x Kisumu/G3 father, 20 offspring); 37-3 (Kisumu x Pimperena, 20 offspring); 45-1 (Mali x Kisumu, 20 offspring); 47-6 (Mali x Kisumu, 20 offspring); 73-2 (Akron x Ghana, 19 offspring); 78-2 (Mali x Kisumu/Ghana, 19 offspring); 80-2 (Kisumu x Akron, 20 offspring).
%
Father colonies with two names, \textit{e.g.} "Kisumu/G3", signify that the father is from one of these two colonies, but exactly which one is unknown.
%
The	colony labels, \textit{e.g.}	"18-5", are	identifiers used for	
each of the crosses	within the project and	have no special meaning.	
%
Information pertaining to the crosses released as part of Ag1000G Phase 1 can be found in the supplementary methods of Ag1000G Consortium (2017) alongside methods for cross creation and processing. \cite{Ag1000gConsortium2017}.
%%

%%%%%%%%%%%%%%%%%%%%%%%%%%%%%%%%%%%%%%%%%%%%%%%%%%%%%%%%%%%%%%%%%%%%%%%%%%%%%%%
\subsection*{2 Whole genome sequencing}

%
Sequencing was performed on the Illumina HiSeq 2000 platform at the Wellcome Trust Sanger Institute.
%
Paired-end multiplex libraries were prepared using the manufacturer's protocol, with the exception that genomic DNA was fragmented using Covaris Adaptive Focused Acoustics rather than nebulization.
%
Multiplexes comprised 12 tagged individual mosquitoes and three lanes of sequencing were generated for each multiplex to even out variations in yield between sequencing runs.
%
Cluster generation and sequencing were undertaken per the manufacturer's protocol for paired-end 100 bp sequence reads with insert size in the range 100-200 bp.
%
Target coverage was 30X per individual.
%%


\subsection*{2.1 Genome accessibility}

%
For various population-genomic analyses, it is necessary to have a map of which positions in the reference genome can be considered accessible (in which we can confidently call genetic variation).
%
For Phase 2 we repeated the Phase 1 acccessibility analyses \cite{Ag1000gConsortium2017} with 1142 samples and the additional Mendelian error information provided by the 11 crosses (in Phase 1 there were four crosses).
%
Following these analyses it was apparent that the Phase 1 accessibility classifications were also appropriate in application to Phase 2.
%
Reference genome positions were accessible if: 
Not repeat masked by DUST; 
No Coverage <= 0.1\% (at most 1 individual had zero coverage); 
Ambiguous Alignment <= 0.1\% (at most 1 individual had ambiguous alignments); 
High Coverage <= 2\% (20 individuals); 
Low Coverage <= 10\% (114 individuals); 
Low Mapping Quality <= 10\% (114 individuals).
%%

\subsection*{2.2 Sequence analysis and variant calling}
%
Sequencing pipelines were unchanged from Phase 1 of the Anopheles 1000 genomes project\cite{Ag1000gConsortium2017}.
%
Briefly, sequence reads were aligned to the AgamP3 reference genome \cite{sharakhova2007update} using \texttt{bwa v0.6.2}, duplicate reads marked \cite{li2009} and SNPs discovered using \texttt{GATK Unified Genotyper 2.7.4} \cite{van2013} following best practice recommendations. 
%%


\subsection*{2.3 Variant Filtering}
%
Following Ag1000G Phase 1 \cite{Ag1000gConsortium2017}, we applied the following SNP filters to reduce the number of false SNP discoveries.
%
We filtered any SNP that occurred at a genome position classified as inaccessible as described in the section on genome accessibility above, thus removing SNPs with evidence for excessively high or low coverage or ambiguous alignment. 
%
We then applied additional filters using variant annotations produced by GATK based on an analysis of Mendelian error in all 11 crosses present in Phase 2 and Ti/Tv ratio similar to that described above for the genome accessibility analysis.
%
We filtered any SNP that failed any of the following criteria: QD <5; FS >100; ReadPosRankSum <-8; BaseQRankSum <-50. 
%%


%
Of 105,486,698 SNPs reported in the raw callset, 57,837,885 passed all quality filters, 13,760,984 (23.8\%) of which were multi-allelic (>= 3 alleles).
%
To produce an analysis-ready VCF file for each chromosome arm, we first removed all non-SNP variants. 
%
We then removed genotype calls for individuals excluded by the sample QC analysis described above, then removed any variants that were no longer variant after excluding individuals. 
%
We then added INFO annotations with genome accessibility metrics and added FILTER annotations per the criteria defined above. 
%
Finally, we added INFO annotations with information about functional consequences of mutations using SNPEFF version 4.1b \cite{Cingolani2012}.
%%


\subsection*{2.4 Sample quality control}
%
A total of 1285 individual mosquitoes were sequenced as part of Ag1000G Phase 2 and included in the cohort for variant discovery. 
%
After variant discovery, quality-control (QC) steps using coverage and contamination filters alongside principal component analysis and meta data concordance were performed to exclude individuals with poor quality sequence and/or genotype data as detailed in \cite{Ag1000gConsortium2017}.
%
A total of 143 individuals were excluded at this stage, retaining 1142 individuals for downstream analyses.

%%


%%%%%%%%%%%%%%%%%%%%%%%%%%%%%%%%%%%%%%%%%%%%%%%%%%%%%%%%%%%%%%%%%%%%%%%%%%%%%%%
\subsection*{2.5 SNP validation}

We used 11 laboratory crosses, each with between 14-20 progeny, to estimate the error associated with genotyping from short reads. 
%
We defined errors at sites where given confident parental genotypes and Mendelian inheritance, the expected genotypes of progeny are known. 
%
At loci where both parents are homozygous for the reference or alternate allele, we expect the progeny to be homozygous for the reference or alternate allele accordingly.
%
When one parent is homozygous for the reference allele, and the other the alternate allele we expect all progeny to be heterozygous. 
%
Therefore we can generate ascertainment error estimates for heterozygous, homozygous reference and homozygous alternate genotypes.
%
To meet the confidence threshold for homozygous calls, both parents must have 30x coverage, and have no discordant reads.
%
Sites are considered erroneously called if at least 10 progeny have genotypes called at that locus, and one or more do not match the expected genotype. 
%
Sites are considered correctly called if at least 10 progeny have genotypes called and all match the expected genotype. 
%
Error in ascertainment rates are computed over each cross, and errors are reported as the median over all 11 crosses. 
%
%
Raw error rates were 0.72\% (0.39, 2.44) for heterozygotes, 0.07\% (0.03, 0.25) for homozygous reference calls, and 0.82\% (0.37, 1.47) for homozygous alternate calls. 
%
Following variant filtering via QC thresholds and the accessibility map these values dropped to 0.26\% (0.15, 1.23) for heterozygotes, 0.02\% (0.01, 0.12) for homozygous reference calls, and 0.80\% (0.31, 1.51) for homozygous alternate calls. 
%%



%%%%%%%%%%%%%%%%%%%%%%%%%%%%%%%%%%%%%%%%%%%%%%%%%%%%%%%%%%%%%%%%%%%%%%%%%%%%%%%
\subsection*{2.6 Haplotype estimation}

%
Haplotype estimation, also known as phasing, was performed on all Phase 2 wild-caught individuals using unchanged methodology from Phase 1 of the Anopheles 1000 genomes project\cite{Ag1000gConsortium2017}.
%
In short, SHAPEIT2 was used to perform statistical phasing with information from sequence reads.
%
Phasing performance was then evaluated by comparison with haplotypes generated from the laboratory crosses and from male X chromosome haplotypes.
%%

%%%%%%%%%%%%%%%%%%%%%%%%%%%%%%%%%%%%%%%%%%%%%%%%%%%%%%%%%%%%%%%%%%%%%%%%%%%%%%%
\subsection*{3 Population structure}

%
Population structure analyses, AIM, F\raisebox{-.4ex}{\scriptsize ST} and PCA were conducted following methods defined in \cite{Ag1000gConsortium2017}.
%
Doubleton sharing was also conducted following \cite{Ag1000gConsortium2017}, except that due low sample sizes in GHgam (12 samples), GNcol (4) and GQgam (9) were excluded.  
%%


%%%%%%%%%%%%%%%%%%%%%%%%%%%%%%%%%%%%%%%%%%%%%%%%%%%%%%%%%%%%%%%%%%%%%%%%%%%%%%%
\subsection*{4 Genetic diversity}

%
Investigations into and utilising genetic diversity, nucleotide diversity, Tajima's D, ROH and IBD (identity by descent), were conducted following methods defined in \cite{Ag1000gConsortium2017} but using the Phase 2 data release of 1,142 samples.
%
In short, scikit-allel ('1.2.0') was used to calculate windowed averages of nucleotide diversity and Tajima's D \cite{miles2018}, IBDseq version r1206 \cite{browning2015} was used to calculate IBD and an	HMM	implemented	in	Python (available in scikit-allel) was used to calculate ROH.

%%%%%%%%%%%%%%%%%%%%%%%%%%%%%%%%%%%%%%%%%%%%%%%%%%%%%%%%%%%%%%%%%%%%%%%%%%%%%%%
\section*{Authors}

%
\textbf{Data analysis group}: Alistair Miles (project lead), Nicholas J. Harding, Giordano Bott\'{a}, Chris S. Clarkson, Tiago Ant\~{a}o, Krzysztof Kozak, Daniel R. Schrider, Andrew D. Kern, Seth Redmond, Igor Sharakhov, Richard D. Pearson, Christina Bergey, Michael C. Fontaine, Martin J. Donnelly, Mara K. N. Lawniczak and Dominic P. Kwiatkowski (chair).
%%

%
\textbf{Partner working group}: Martin J. Donnelly (chair), Diego Ayala, Nora J. Besansky, Austin Burt, Beniamino Caputo, Alessandra della Torre, Michael C. Fontaine, H. Charles J. Godfray, Matthew W. Hahn, Andrew D. Kern, Dominic P. Kwiatkowski, Mara K. N. Lawniczak, Janet Midega, Daniel E. Neafsey, Samantha O'Loughlin, Jo\~{a}o Pinto, Michelle M. Riehle, Igor Sharakhov, Kenneth D. Vernick, David Weetman, Craig S. Wilding and Bradley J. White.
%%

%
\textbf{Sample collections}: Angola: Arlete D. Troco, Jo\~{a}o Pinto; Burkina Faso: Abdoulaye Diabat\'{e}, Samantha O'Loughlin, Austin Burt; Cameroon: Carlo Costantini , Kyanne R. Rohatgi, Nora J. Besansky; Equatorial Guinea: Jorge Cano; Gabon: Nohal Elissa, Jo\~{a}o Pinto; The Gambia: Davis C. Nwakanma, Musa Jawara; Guinea: Boubacar Coulibaly, Michelle M. Riehle, Kenneth D. Vernick; Guinea-Bissau: Jo\~{a}o Pinto, Jo\~{a}o Dinis; Kenya: Janet Midega, Charles Mbogo, Philip Bejon; Mayotte: Gilbert Le Goff, Vincent Robert; Uganda: Craig S. Wilding, David Weetman, Henry D. Mawejje, Martin J. Donnelly; Crosses: David Weetman, Craig S. Wilding, Martin J. Donnelly.
%%

%
\textbf{Sequencing and data production}: Jim Stalker, Kirk Rockett, Eleanor Drury, Daniel Mead, Anna Jeffreys, Christina Hubbart, Kate Rowlands, Alison T. Isaacs, Dushyanth Jyothi, Cinzia Malangone and Maryam Kamali.
%%

%
\textbf{Web application development}: Paul Vauterin, Ben Jeffrey, Ian Wright, Lee Hart and Krzysztof Kluczy\'{n}ski.
%%

%
\textbf{Project coordination}: Victoria Cornelius, Bronwyn MacInnis, Christa Henrichs, Rachel Giacomantonio and Dominic P. Kwiatkowski.



%%%%%%%%%%%%%%%%%%%%%%%%%%%%%%%%%%%%%%%%%%%%%%%%%%%%%%%%%%%%%%%%%%%%%%%%%%%%%%%
%%%%%%%%%%%%%%%%%%%%%%%%%%%%%%%%%%%%%%%%%%%%%%%%%%%%%%%%%%%%%%%%%%%%%%%%%%%%%%%
% TODO enable bibliography
%\printbibliography




\beginsupplement
%%%%%%%%%%%%%%%%%%%%%%%%%%%%%%%%%%%%%%%%%%%%%%%%%%%%%%%%%%%%%%%%%%%%%%%%%%%%%%%
%%%%%%%%%%%%%%%%%%%%%%%%%%%%%%%%%%%%%%%%%%%%%%%%%%%%%%%%%%%%%%%%%%%%%%%%%%%%%%%

\clearpage

\section*{Supplementary figures and tables}


\clearpage

\begin{figure}[H]
	\begin{center}
		\includegraphics*[width=6.3in]{artwork/AIM_figure_scaled.jpg}
	\end{center}
	\caption{Ancestry informative markers (AIM). Rows represent individual mosquitoes (grouped by population) and columns represent SNPs (grouped by chromosome arm). Colours represent species genotype. The column at the far left shows the species assignment according to the conventional molecular test based on a single marker on the X chromosome, which was performed for all populations except Guinea Bissau (GW), The Gambia (GM) and Kenya (KE). The column at the far right shows the genotype for kdr variants in \textit{Vgsc} codon 995. Lines at the lower edge show the physical locations of the AIM SNPs.}
	\label{aim}
\end{figure}



\clearpage


\begin{figure}[H]
	\begin{center}
		\includegraphics*[width=6.3in]{artwork/Watterson_theta.jpeg}
	\end{center}
	\caption{@@}
	\label{div}
\end{figure}

\clearpage


\begin{figure}[H]
	\begin{center}
		\includegraphics*[width=6.3in]{artwork/sfs_supplementary_large.jpg}
	\end{center}
	\caption{@@}
	\label{div}
\end{figure}

\clearpage

\begin{figure}[H]
	\begin{center}
		\includegraphics*[width=6.3in]{notebooks/refdiff/refdiff_phase2_combined.jpg}
	\end{center}
	\caption{Divergence from the AgamP4 reference genome, calculated as \textit{Dxy}, is largely similar for \textit{An. coluzzii} and \textit{An. gambiae}, with the exception of the centromere of chromosome X (a). Comparing three populations of \textit{An. coluzzii} (b) or \textit{An. gambiae} (c) highlights the strong effect of the 2La chromosomal inversion on the accumulation of genetic variation.}
	\label{refdiff}
\end{figure}



\clearpage

\begin{figure}[H]
    \textbf{a}
	\begin{center}
        \includegraphics[width=1.0\linewidth]{artwork/dsx_a.pdf}
	\end{center}
    \textbf{b}
	\begin{center}
        \includegraphics[width=1.0\linewidth]{artwork/dsx_b.pdf}
	\end{center}
    \caption{%
    Nucleotide diversity within the female-specific exon 5 of the doublesex gene (\textit{dsx}; AGAP004050), a key component of the sex determination pathway and a gene targeted for Cas9-based homing endonuclease gene drive.
    %
    In both plots, the location of exon 5 within the female-specific isoform (AGAP004050-RB) is shown above (black = coding sequence; grey = untranslated region), with additional annotations to show the location of the Cas9 target sequence described in \cite{kyrou2018}, and the putative exon splice enhancing sequences (``RE'') reported in @@REF.
    %
    The main region of the plot shows nucleotide diversity averaged across all Ag1000G Phase 2 populations, computed in 23 bp moving windows.
    %
    Regions shaded pale red indicate regions not accessible to SNP calling.
    %
    Triangle markers below show the locations of SNPs discovered in Ag1000G Phase 2 (green = passed variant filters; red = failed variant filters).
    %
    \textbf{a}, exon5/intron4 boundary.
    %
    \textbf{b}, exon5/intron6 boundary.
}
    \label{fig:dsx}
\end{figure}


%% Table S1 - crosses.
%
\afterpage{%
%\clearpage
% N.B., for some reason using \newgeometry causes page number to get dropped from the subsequent page, so disable for now - not needed if using \footnotesize.
\newgeometry{margin=2cm}
\thispagestyle{empty}
\begin{table}[h]
  \normalsize
  \centering
  \begin{threeparttable}

  \caption{
%
\textbf{colony crosses}.
}
  \label{table:colony crosses}
  
\begin{tabular}{lllr}
\toprule
Cross ID & 
Mother Colony & 
Father Colony &
N progeny \\
\midrule

18-5 & Ghana & Kisumu/G3 & 20 \\

29-2 & Ghana & Kisumu & 20 \\

36-9 & Ghana & Mali & 20 \\

37-3 & Kisumu & Pimperena & 20 \\

42-4 & Mali & Kisumu/Ghana & 14 \\

45-1 & Mali & Kisumu & 20 \\

46-9 & Pimperena & Mali & 20 \\

47-6 & Mali & Kisumu & 20 \\

73-2 & Akron & Ghana & 19 \\

78-2 & Mali & Kisumu/Ghana & 19 \\

80-2 & Kisumu & Akron & 20 \\

\bottomrule
\end{tabular}

  \end{threeparttable}

\end{table}
\restoregeometry
} % end afterpage
%% end Table 1

\clearpage


\end{document}
