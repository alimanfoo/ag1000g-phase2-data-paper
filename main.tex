\documentclass[a4paper,11pt,abstracton,hidelinks]{scrartcl}

\usepackage[margin=3cm]{geometry}
\usepackage{graphicx}
\usepackage[UKenglish]{babel}
\usepackage{csquotes}
\usepackage[style=numeric,citestyle=numeric,backend=biber,sorting=none,doi=false,url=false]{biblatex}
\usepackage{float}
\usepackage[export]{adjustbox}
\usepackage[T1]{fontenc}
\usepackage{lmodern}
\usepackage[textsize=tiny]{todonotes}
\usepackage[labelsep=period,font=small,labelfont=bf,format=plain]{caption}
\captionsetup[table]{
  position=above,
  belowskip=10pt,
  aboveskip=0pt,
}
\usepackage[group-separator={,}]{siunitx}
\usepackage{booktabs}
\usepackage{pdflscape}
\usepackage{tablefootnote}
\usepackage{authblk}
\usepackage{threeparttable}
\usepackage{afterpage}
\usepackage{lineno}
\linenumbers
\usepackage{setspace}
\usepackage{hyperref}
\doublespacing

\newcommand{\beginsupplement}{%
  \setcounter{table}{0}
  \renewcommand{\thetable}{S\arabic{table}}%
  \setcounter{figure}{0}
  \renewcommand{\thefigure}{S\arabic{figure}}%
}


% TODO create refs.bib
\addbibresource{refs.bib}


\title{
Nucleotide variation in 1,142 genomes of the African malaria vectors \emph{Anopheles gambiae} and \emph{Anopheles coluzzii}
}



\author[1]{\small The \emph{Anopheles gambiae} 1000 Genomes Consortium}
\affil[1]{\footnotesize A list of consortium members appears at the end of the paper}

\begin{document}

\maketitle


%%%%%%%%%%%%%%%%%%%%%%%%%%%%%%%%%%%%%%%%%%%%%%%%%%%%%%%%%%%%%%%%%%%%%%%%%%%%%%%
%%%%%%%%%%%%%%%%%%%%%%%%%%%%%%%%%%%%%%%%%%%%%%%%%%%%%%%%%%%%%%%%%%%%%%%%%%%%%%%
\begin{abstract}

%%
TODO
%%

\end{abstract}


%%%%%%%%%%%%%%%%%%%%%%%%%%%%%%%%%%%%%%%%%%%%%%%%%%%%%%%%%%%%%%%%%%%%%%%%%%%%%%%
%%%%%%%%%%%%%%%%%%%%%%%%%%%%%%%%%%%%%%%%%%%%%%%%%%%%%%%%%%%%%%%%%%%%%%%%%%%%%%%
\section*{Introduction}

%%
TODO
%%


%%%%%%%%%%%%%%%%%%%%%%%%%%%%%%%%%%%%%%%%%%%%%%%%%%%%%%%%%%%%%%%%%%%%%%%%%%%%%%%
%%%%%%%%%%%%%%%%%%%%%%%%%%%%%%%%%%%%%%%%%%%%%%%%%%%%%%%%%%%%%%%%%%%%%%%%%%%%%%%
\section*{Results}

%% 
TODO
%%

%% Population sampling
\subsection*{Population Sampling}


%
In Phase 2 of the AG1000G Project we sequenced the genomes from 655 \textit{Anopheles gambiae} individuals, 283 \textit{An. coluzzii} and 204 individuals of "unknown" species (individuals with apparent mixed or unclear ancestry), 1142 individuals in total.
%
The project now spans 16 populations of wild-caught mosquitoes, sampled from 13 countries across Sub-Saharan Africa (Figure 1; Table 1). 
%
Single nucleotide polymorphism (SNPs) were identified after read alignment to the AgamP3 reference genome \cite{Holt2002}.
%
Accessible regions of the genome are those in which we can confidently call SNPs which can then be used in analysis of population variation.
%
Following analysis we defined 61\% (140Mbp) of the genome as accessible, including 91\% (18Mbp) of the exome and 58\% (121Mbp) of non-coding positions.
%
From these accessible regions of genome we identified 57,837,885 high-quality SNPs, of which 24\% were classed as multi-allelic (three or more alleles), generating an average of one variant allele every 1.9 bases of accessible genome (increasing from one in every 2.2 bases in Phase 1 \cite{Ag1000gConsortium2017}). 
%
The addition of 377 Phase 2 samples added over five million new SNPs to the AG1000G data set.
%%

\begin{figure}[h]
	\begin{center}
		\includegraphics*[width=5.8in]{artwork/collection_site_map.jpg}
	\end{center}
	\caption{AG1000G Phase 2 sampling locations. Colour of circle denotes species collected at location and area represents sample size. Colours on the map represent ecosystem classes; dark green designates forest ecosystems; see figure 9 in \cite{sayre2013} for a compete colour legend.}
	\label{sample_map}
\end{figure}


%% Table 1 - Population sampling.
%
\afterpage{%
%\clearpage
% N.B., for some reason using \newgeometry causes page number to get dropped from the subsequent page, so disable for now - not needed if using \footnotesize.
\newgeometry{margin=2cm}
\begin{landscape}
\thispagestyle{empty}
\begin{table}[h]
  \footnotesize
  \centering
  \begin{threeparttable}

  \caption{
%
\textbf{AG1000G Phase 2 sampling locations}.
}

  \label{table:sampling_locations}

  
\begin{tabular}{lllcccccc}
\toprule
\multicolumn{6}{c}{\textbf{Collection}} &
\multicolumn{3}{c}{\textbf{\emph{Anopheles} species counts}}\\
\cmidrule(r){1-6}
\cmidrule(r){7-9}
Country & 
Location & 
Site &
Year &
Latitude & 
Longitude & 
\emph{gambiae} & 
\emph{coluzzii} & 
Unknown\\
\midrule

Angola & Luanda &  & 2009 & -8.8210 & 13.2910 & 0 & 78 & 0 \\

Burkina Faso & Bana &  & 2012 & 11.2330 & -4.4720 & 20 & 40 & 0 \\

 & Pala &  & 2012 & 11.1500 & -4.2350 & 46 & 10 & 0 \\

 & Souroukoudinga &  & 2012 & 11.2350 & -4.5350 & 26 & 25 & 0 \\

Cameroon & Daiguene &  & 2009 & 4.7770 & 13.8440 & 96 & 0 & 0 \\

 & Gado Badzere &  & 2009 & 5.7470 & 14.4420 & 73 & 0 & 0 \\

 & Mayos &  & 2009 & 4.3410 & 13.5580 & 105 & 0 & 0 \\

 & Zembe Borongo &  & 2009 & 5.7470 & 14.4420 & 23 & 0 & 0 \\

Cote d'Ivoire & Tiassale &  & 2012 & 5.8984 & -4.8229 & 0 & 71 & 0 \\

Equatorial Guinea & Bioko &  & 2002 & 3.7000 & 8.7000 & 9 & 0 & 0 \\

France & Mayotte & Bouyouni & 2011 & -12.7378 & 45.1417 & 1 & 0 & 0 \\

 &  & Combani & 2011 & -12.7787 & 45.1429 & 5 & 0 & 0 \\

 &  & Karihani Lake & 2011 & -12.7965 & 45.1217 & 3 & 0 & 0 \\

 &  & Mont Benara & 2011 & -12.8570 & 45.1552 & 2 & 0 & 0 \\

 &  & Mtsamboro Forest Reserve & 2011 & -12.7027 & 45.0811 & 1 & 0 & 0 \\

 &  & Mtsanga Charifou & 2011 & -12.9907 & 45.1557 & 8 & 0 & 0 \\

 &  & Sada & 2011 & -12.8521 & 45.1039 & 4 & 0 & 0 \\

Gabon & Libreville &  & 2000 & 0.3840 & 9.4550 & 69 & 0 & 0 \\

Gambia, The & Njabakunda & Kerr Birom Kardo & 2011 & 13.5500 & -15.9000 & 0 & 0 & 19 \\

 &  & Kerr Sama Kuma & 2011 & 13.5500 & -15.9000 & 0 & 0 & 8 \\

 &  & Maria Samba Nyado & 2011 & 13.5500 & -15.9000 & 0 & 0 & 18 \\

 &  & Sare Illo Buya & 2011 & 13.5500 & -15.9000 & 0 & 0 & 20 \\

Ghana & Koforidua &  & 2012 & 6.0945 & -0.2609 & 0 & 1 & 0 \\

 & Madina &  & 2012 & 5.6685 & -0.2193 & 12 & 12 & 0 \\

 & Takoradi &  & 2012 & 4.9122 & -1.7740 & 0 & 20 & 0 \\

 & Twifo Praso &  & 2012 & 5.6086 & -1.5493 & 0 & 22 & 0 \\

Guinea & Koraboh &  & 2012 & 9.2500 & -9.9170 & 22 & 0 & 0 \\

 & Koundara &  & 2012 & 8.5000 & -9.4170 & 18 & 4 & 0 \\

Guinea-Bissau & Antula &  & 2010 & 11.8910 & -15.5820 & 0 & 0 & 58 \\

 & Safim &  & 2010 & 11.9569 & -15.6492 & 0 & 0 & 33 \\

Kenya & Kilifi & Junju & 2012 & -3.8620 & 39.7450 & 0 & 0 & 16 \\

 &  & Mbogolo & 2012 & -3.6350 & 39.8580 & 0 & 0 & 32 \\

Uganda & Tororo & Nagongera & 2012 & 0.7700 & 34.0260 & 112 & 0 & 0 \\

\bottomrule
\end{tabular}



  \end{threeparttable}

\end{table}
\end{landscape}
\restoregeometry
} % end afterpage
%% end Table 1


%% Difference from reference
\subsection*{AgamP4 reference genome is equally similar to both species in the study}

The genomes of \textit{An. coluzzii} and \textit{An. gambiae} samples were similarly diverged from the reference genome used for alignment, AgamP4 \cite{Holt2002}, indicating that comparisons can be made between the two species based on our results (Fig. \ref{refdiff}a).
%
The similarity in levels of divergence reflect the mixed ancestry of the PEST strain from which the reference genome was derived \cite{Holt2002}.
%
An exception to this is the pericentromeric region of chromosome X, a known genomic region of divergence between the two species \cite{Ag1000gConsortium2017}, where the reference genome is more similar to \textit{An. coluzzii} than to \textit{An. gambiae}.
%
This similarity to \textit{An. coluzzii} on the X chromosome may be due to the history of artificial selection for the X-linked pink eye mutation in the reference strain \cite{Holt2002}, as this originated in the \textit{An. coluzzii} parent it may have led to the removal of any \textit{An. gambiae} ancestry in this region.
%
Reductions in divergence from the reference genome in regions of low recombination are apparent in the centromeres and telomeres, as well as in the large 2La polymorphic inversion in both species \cite{coluzzi2002} (Figure \ref{refdiff}b \& c). 
%
All analyses of geographical population structure were conducted on euchromatic regions of Chromosome 3, which avoids regions of polymorphic inversions, reduced recombination and unequal divergence from the reference genome \cite{Ag1000gConsortium2017}.

\begin{figure}[h](sister species)
	\begin{center}
		\includegraphics*[width=6.3in]{notebooks/refdiff/refdiff_phase2_combined.jpg}
	\end{center}
	\caption{Divergence from the AgamP4 reference genome, calculated as \textit{Dxy}, is largely similar for \textit{An. coluzzii} and \textit{An. gambiae}, with the exception of the centromere of chromosome X (a). Comparing three populations of \textit{An. coluzzii} (b) or \textit{An. gambiae} (c) highlights the strong effect of the 2La chromosomal inversion on the accumulation of genetic variation.}
	\label{refdiff}
\end{figure}




%% Population Structure
\subsection*{Population structure}

Analysis of population structure in the nine AG1000G Phase 1 populations revealed medically important demographic features such a levels of differentiation between populations and movement of insecticide resistance mutations across Sub-Saharan Africa \cite{Ag1000gConsortium2017}.
% 
Here we revisit population structure analyses to include the seven populations novel to Phase 2.
%%

\begin{figure}[H]
	\begin{center}
		\includegraphics*[width=6.3in]{artwork/pca_3L_main_text.jpeg}
	\end{center}
	\caption{Principal component analysis of the 1142 wild-caught mosquitoes derived from nucleotide variation in the accessible region of chromosome 3L.}
	\label{pca}
\end{figure}

%
Of particular importance to the Phase 2 data release are the three \textit{An. coluzzii} populations, Cote d'Ivoire (CIcol), Guinea (GNcol) and Ghana (GHcol) (Figure \ref{sample_map}). 
%
In Phase 1, just two of the nine populations belonged to this species and these, Burkina Faso (BFcol) and Angola (AOcol), appeared quite different to each other in population structure (principal component analysis (PCA), doubleton sharing and F\raisebox{-.4ex}{\scriptsize ST}).
%
Consequently, making general conclusions about the history and population structure of this major malaria vector was problematic and the relationship to its sister species \textit{An. gambiae} difficult to investigate \cite{Ag1000gConsortium2017}.
%
Less population structure is found between the the three novel Phase 2 \textit{An. coluzzii} populations than between the Phase 1 \textit{An. coluzzii} populations.
%
PCA analyses found CIcol, GNcol and GHcol clustering together in initial PCA components, the Phase 1 population of BFcol clusters closely but is separated by PC1, with AOcol distant by both PC1 and PC2 (Figure \ref{pca}).
%
Despite the PC1 separation of BFcol from the new \textit{An. coluzzii} samples, doubleton sharing and F\raisebox{-.4ex}{\scriptsize ST}) analyses
reveal high sharing and weak allelic differentiation respectively between all \textit{An. coluzzii} populations with the exception of AOcol (Figure \ref{doubletons}; Figure \ref{fst}).
%%

\begin{figure}[H]
	\begin{center}
		\includegraphics*[width=4.3in]{artwork/doubletons.jpeg}
	\end{center}
	\caption{Allele sharing in doubleton (\textit{f2}) variants. The height of the coloured bars represent the probability of sharing a doubleton allele between two populations. Heights are normalized row-wise for each population.}
	\label{doubletons}
\end{figure}

\begin{figure}[H]
	\begin{center}
		\includegraphics*[width=5.3in]{artwork/pairwise_fst.jpeg}
	\end{center}
	\caption{Average allele frequency differentiation (F\raisebox{-.4ex}{\scriptsize ST}) between pairs of populations. The bottom left triangle shows average F\raisebox{-.4ex}{\scriptsize ST} values between each population pair. The top right triangle shows the Z score for each F\raisebox{-.4ex}{\scriptsize ST} value estimated via a block-jackknife procedure.}
	\label{fst}
\end{figure}



%
The Phase 2 Gambian (GM) population is strikingly similar in multiple aspects to the the samples from neighbouring Guinea Bissau (GW).
%
Both populations have highly 'hybridised' autosomes (Figure \ref{aim}), fall together in early principal components (Figure \ref{pca}), show high doubleton sharing with each other but not between other populations (Figure \ref{doubletons}) and are weakly differentiated (F\raisebox{-.4ex}{\scriptsize ST} = 0.007 Figure \ref{fst}).
%
The Phase 2 cohort also includes two \textit{An. gambiae} populations collected from islands, Bioko (GQgam) off the west coast of Africa and Mayotte (FRgam) off the eastern seaboard; despite their island collection our analyses found them highly dissimilar. 
%
GQgam clusters with Ugandan (UGcol), Ghanaian (GHcol) and Cameroonian (CMcol) \textit{An. gambiae} in early PCA components (Figure \ref{pca}), though has elevated levels of genetic differentiation when compared to other 
\textit{An. gambiae vs. An. gambiae} F\raisebox{-.4ex}{\scriptsize ST} values (exluding FRgam) (Figure \ref{fst}). 
%
FRgam, however, is an outlier in principal components three and four, has almost no doubleton sharing with any other populations and the only population more genetically differentiated in pairwise comparisons is Kenya (KE) which known to be highly bottlenecked \cite{Ag1000gConsortium2017} (Figure \ref{fst}).
% 
Unlike KE, which appears 'hybridised' in AIM analysis, FRgam is comparable to other An. gambiae populations (Figure \ref{aim}).
%%


\begin{figure}[H]
	\begin{center}
		\includegraphics*[width=6.3in]{artwork/AIM_figure_scaled.jpg}
	\end{center}
	\caption{Ancestry informative markers (AIM). Rows represent individual mosquitoes (grouped by population) and columns represent SNPs (grouped by chromosome arm). Colours represent species genotype. The column at the far left shows the species assignment according to the conventional molecular test based on a single marker on the X chromosome, which was performed for all populations except Guinea Bissau (GW), The Gambia (GM) and Kenya (KE). The column at the far right shows the genotype for kdr variants in \textit{Vgsc} codon 995. Lines at the lower edge show the physical locations of the AIM SNPs.}
	\label{aim}
\end{figure}


%%%%%%%%%%%%%%%%%%%%%%%%%%%%%%%%%%%%%%%%%%%%%%%%%%%%%%%%%%%%%%%%%%%%%%%%%%%%%%%
%%%%%%%%%%%%%%%%%%%%%%%%%%%%%%%%%%%%%%%%%%%%%%%%%%%%%%%%%%%%%%%%%%%%%%%%%%%%%%%
\section*{Discussion}

%%
TODO
%%


%%%%%%%%%%%%%%%%%%%%%%%%%%%%%%%%%%%%%%%%%%%%%%%%%%%%%%%%%%%%%%%%%%%%%%%%%%%%%%%
%%%%%%%%%%%%%%%%%%%%%%%%%%%%%%%%%%%%%%%%%%%%%%%%%%%%%%%%%%%%%%%%%%%%%%%%%%%%%%%
\section*{Methods}


%%%%%%%%%%%%%%%%%%%%%%%%%%%%%%%%%%%%%%%%%%%%%%%%%%%%%%%%%%%%%%%%%%%%%%%%%%%%%%%
\subsection*{1. Population sampling}

%
Ag1000g Phase 2 mosquitoes were collected from natural populations at 33 sites from 23
geographical locations in 13 sub-Saharan African countries. 
%
Samples have been grouped into 16 populations based on collection country and species (Figure \ref{sample_map} \& Table \ref{table:sampling_locations}).
%
Throughout, we use species nomenclature following Coetzee \textit{et al}. \cite{Coetzee2013};	
%
prior to	 Coetzee	 \textit{et al}., \textit{An. gambiae} was known as \textit{An. gambiae sensu stricto} (S form) and \textit{An. coluzzii} was known as \textit{An. gambiae sensu stricto} (M form).
%
Details of the eighteen collection sites novel to Ag1000g Phase 2 (dates, collection and DNA extraction methods) can be found in the Methods section.
%
Information pertaining to the collection of samples released as part of Ag1000g Phase 1 can be found in the supplementary methods of Ag1000g Consortium (2017) \cite{Ag1000gConsortium2017}.
%
Unless otherwise stated, the DNA extraction method used for the collections described below was Qiagen DNeasy Blood and Tissue Kit (Qiagen Science, MD, USA).
%%

%%%%%%%%%%%%%%%%%%%%%%%%%%%%%%%%%%%%%%%%%%%%%%%%%%%%%%%%%%%%%%%%%%%%%%%%%%%%%%%
\subsection*{1.1 Novel Phase 2 natural populations}
%
\textbf{C\^{o}te d'Ivoire (CIcol)}: Tiassal\'{e} (-4.823, 5.898) is located in the evergreen forest zone of southern C\^{o}te d'Ivoire.
%
The primary agricultural activity is rice cultivation in irrigated fields.
%
High malaria transmission occurs during the rainy seasons, between May and November.
%
Samples were collected as larvae from irrigated rice fields by dipping between May and Septermber 2012.
%
All larvae were reared to adults and females preserved over silica for DNA extraction.
%
Specimens from this site were all \textit{An. coluzzii}, determined by PCR assay \cite{Santolamazza2008}
%%

\textbf{Bioko Island - Equatorial Guinea (GQgam)}: Collections were performed during the rainy season in September, 2002 by overnight CDC light traps in Sacriba of Bioko island (8.7, 3.7).
%
Specimens were stored dry on silica gel before DNA extraction.
%
Specimens contributed from this site were \textit{An. gambiae} females, genotype determined by two assays \cite{Scott1993, Santolamazza2004}.
%
All specimens had the 2L\textsuperscript{+a}/2L\textsuperscript{+a} karyotype as determined by the molecular PCR diagnostics \cite{White2007}. 
%
These mosquitoes represent a population that inhabited Bioko Island before a comprehensive malaria control intervention initiated in February 2004 \cite{Sharp2007}. 
%
After the intervention \textit{An. gambiae} was declining, and more recently almost only \textit{An. coluzzii} can be found \cite{Overgaard2012}.
%%

%
\textbf{Mayotte Island - France (FRgam)}: Samples were collected as larvae during March-April 2011 in temporary pools by dipping, in Bouyouni (-12.738, 45.143), M'Tsamboro Forest Reserve (-12.703, 45.081), Combani (-12.779, 45.143), Mtsanga Charifou (-12.991, 45.156), Karihani Lake forest reserve (-12.797, 45.122), and Sada (-12.852, 45.104) in Mayotte island.
%
Larvae were stored in 80\% ethanol prior to DNA extraction. 
%
All specimens contributed to Ag1000G Phase 2 were \textit{An. gambiae} \cite{Santolamazza2004} with the standard 2L\textsuperscript{+a}/2L\textsuperscript{+a} or inverted 2L\textsuperscript{a}/2L\textsuperscript{a} karyotype as determined by the molecular PCR diagnostics \cite{White2007}.
%
The samples were identified as males or females by the sequencing read coverage of the X chromosome using LookSeq \cite{Manske2009}.
%%

%
\textbf{The Gambia (GM)}: Indoor resting female mosquitoes were collected by pyrethrum spray catch from four hamlets around Njabakunda (-15.90, 13.55), North Bank Region, The Gambia between August and October 2011.
%
The four hamlets were Maria Samba Nyado, Sare Illo Buya, Kerr Birom Kardo, and Kerr Sama Kuma; all are within 1 km of each other.
%
This is an area of unusually high hybridization rates between \textit{An. gambiae s.s.} and \textit{An. coluzzii} \cite{Caputo2008, Nwakanma2013}.
%
Njabakunda village is approximately 30 km to the west of Farafenni town and 4 km away from the Gambia River.
%
The vegetation is a mix of open savannah woodland and farmland.
%
With apparent high gene-flow in the region, it is problematic to assign species to these samples.
%%

%
\textbf{Ghana (GHcol/GHgam)}: Twifo Praso (5.609, -1.549) a peri-urban community located in semi-deciduous forest in the Central Region of Ghana.
%
It is an extensive agricultural area characterised by small-scale (vegetable growing) and large-scale commercial farms such as oil palm and cocoa plantations.
%
Mosquito samples were collected as larvae from puddles near farms between September and October, 2012.
%
Madina (5.668,	-0.219) is suburb of Accra within a coastal savanna zone of Ghana. 
%
It is an urban community characterised by myriad vegetable-growing areas.
%
The vegetation consists of mainly grassland interspersed with dense short thickets often less than 5 m high with a few trees.
%
Specimens were sampled from puddles near roadsides and farms between October and December 2012.
%
Takoradi (4.912, -1.774) is the capital city of Western Region of Ghana.
%
It is an urban community located in the coastal savanna zone.
%
Mosquito samples were collected from puddles near road construction and farms between August and September 2012.
%
Koforidua (6.094, -0.261) is a capital city of Eastern Region of Ghana and is located in semi-deciduous forest. 
%
It is an urban community characterized by numerous small-scale vegetable farms. 
%
Samples were collected from puddles near road construction and farms between August and September 2012.
%
Larvae from all collection sites were reared to adults and females preserved over silica for DNA extraction.
%
Both \textit{An. gambiae} and \textit{An. coluzzii} were collected from these sites, determined by PCR assay \cite{Santolamazza2008}.
%%

%
\textbf{Guinea-Bissau (GW)}: Mosquitoes were collected in October 2010 using indoor CDC light traps, in the village of Safim (11.957, -15.649), ca. 11 km north of Bissau city, the capital of the country.
%
Malaria is hyperendemic in the region and transmitted by members of the Anopheles gambiae complex (Vicente et al., 2017).
%
\textit{Anopheles arabiensis, An. melas, An. coluzzii} and \textit{An. gambiae}, as well as hybrids between the latter two species, are known to occur in the region \cite{Gordicho2014, Vicente2017}.
%
Mosquitoes were preserved individually on 0.5ml micro-tubes filled with silica gel and cotton. DNA extraction was performed by a phenol-chloroform protocol \cite{Donnelly1999}.
%
Guinea-Bissau is another region where defining species is problematic (Vicente), so no species has been assigned here.
%%

%%%%%%%%%%%%%%%%%%%%%%%%%%%%%%%%%%%%%%%%%%%%%%%%%%%%%%%%%%%%%%%%%%%%%%%%%%%%%%%
\subsection*{1.2 Colony populations}

%%
TODO
%%

%%%%%%%%%%%%%%%%%%%%%%%%%%%%%%%%%%%%%%%%%%%%%%%%%%%%%%%%%%%%%%%%%%%%%%%%%%%%%%%
\subsection*{2 Whole genome sequencing}

%
Sequencing was performed on the Illumina HiSeq 2000 platform at the Wellcome Trust Sanger Institute.
%
Paired-end multiplex libraries were prepared using the manufacturer's protocol, with the exception that genomic DNA was fragmented using Covaris Adaptive Focused Acoustics rather than nebulization.
%
Multiplexes comprised 12 tagged individual mosquitoes and three lanes of sequencing were generated for each multiplex to even out variations in yield between sequencing runs.
%
Cluster generation and sequencing were undertaken per the manufacturer's protocol for paired-end 100 bp sequence reads with insert size in the range 100-200 bp.
%
Target coverage was 30X per individual.
%%


\subsection*{2.1 Genome accessibility}

%
For various population-genomic analyses, it is necessary to have a map of which positions in the reference genome can be considered accessible (in which we can confidently call genetic variation).
%
For Phase 2 we repeated the Phase 1 acccessibility analyses \cite{Ag1000gConsortium2017} with 1142 samples and the additional Mendelian error information provided by the 11 crosses (in Phase 1 there were four crosses).
%
Following these analyses it was apparent that the Phase 1 accessibility classifications were also appropriate in application to Phase 2.
%
Reference genome positions were accessible if: 
Not repeat masked by DUST; 
No Coverage <= 0.1\% (at most 1 individual had zero coverage); 
Ambiguous Alignment <= 0.1\% (at most 1 individual had ambiguous alignments); 
High Coverage <= 2\% (20 individuals); 
Low Coverage <= 10\% (114 individuals); 
Low Mapping Quality <= 10\% (114 individuals).
%%

\subsection*{2.2 Sequence analysis and variant calling}
%
Sequencing pipelines were unchanged from Phase 1 of the Anopheles 1000 genomes project\cite{Ag1000gConsortium2017}.
%
Briefly, sequence reads were aligned to the AgamP3 reference genome \cite{sharakhova2007update} using \texttt{bwa v0.6.2}, duplicate reads marked \cite{li2009} and SNPs discovered using \texttt{GATK Unified Genotyper 2.7.4} \cite{van2013} following best practice recommendations. 
%%


\subsection*{2.3 Variant Filtering}
%
Following AG1000G Phase 1 \cite{Ag1000gConsortium2017}, we applied the following SNP filters to reduce the number of false SNP discoveries.
%
We filtered any SNP that occurred at a genome position classified as inaccessible as described in the section on genome accessibility above, thus removing SNPs with evidence for excessively high or low coverage or ambiguous alignment. 
%
We then applied additional filters using variant annotations produced by GATK based on an analysis of Mendelian error in all 11 crosses present in Phase 2 and Ti/Tv ratio similar to that described above for the genome accessibility analysis.
%
We filtered any SNP that failed any of the following criteria: QD <5; FS >100; ReadPosRankSum <-8; BaseQRankSum <-50. 
%%


%
Of 105,486,698 SNPs reported in the raw callset, 57,837,885 passed all quality filters, 13,760,984 (23.8\%) of which were multi-allelic (>= 3 alleles).
%
To produce an analysis-ready VCF file for each chromosome arm, we first removed all non-SNP variants. 
%
We then removed genotype calls for individuals excluded by the sample QC analysis described above, then removed any variants that were no longer variant after excluding individuals. 
%
We then added INFO annotations with genome accessibility metrics and added FILTER annotations per the criteria defined above. 
%
Finally, we added INFO annotations with information about functional consequences of mutations using SNPEFF version 4.1b \cite{Cingolani2012}.
%%


\subsection*{2.4 Sample quality control}
%
A total of 1285 individual mosquitoes were sequenced as part of Ag1000G phase 2 and included in the cohort for variant discovery. 
%
After variant discovery, quality-control (QC) steps using coverage and contamination filters alongside principal component analysis and meta data concordance were performed to exclude individuals with poor quality sequence and/or genotype data as detailed in \cite{Ag1000gConsortium2017}.
%
A total of 143 individuals were excluded at this stage, retaining 1142 individuals for downstream analyses.

%%


%%%%%%%%%%%%%%%%%%%%%%%%%%%%%%%%%%%%%%%%%%%%%%%%%%%%%%%%%%%%%%%%%%%%%%%%%%%%%%%
\subsection*{2.5 SNP validation}

We used 11 laboratory crosses, each with between 14-20 progeny, to estimate the error associated with genotyping from short reads. 
%
We defined errors at sites where given confident parental genotypes and Mendelian inheritance, the expected genotypes of progeny are known. 
%
At loci where both parents are homozygous for the reference or alternate allele, we expect the progeny to be homozygous for the reference or alternate allele accordingly.
%
When one parent is homozygous for the reference allele, and the other the alternate allele we expect all progeny to be heterozygous. 
%
Therefore we can generate ascertainment error estimates for heterozygous, homozygous reference and homozygous alternate genotypes.
%
To meet the confidence threshold for homozygous calls, both parents must have 30x coverage, and have no discordant reads.
%
Sites are considered erroneously called if at least 10 progeny have genotypes called at that locus, and one or more do not match the expected genotype. 
%
Sites are considered correctly called if at least 10 progeny have genotypes called and all match the expected genotype. 
%
Error in ascertainment rates are computed over each cross, and errors are reported as the median over all 11 crosses. 
%
%
Raw error rates were 0.72\% (0.39, 2.44) for heterozygotes, 0.07\% (0.03, 0.25) for homozygous reference calls, and 0.82\% (0.37, 1.47) for homozygous alternate calls. 
%
Following variant filtering via QC thresholds and the accessibility map these values dropped to 0.26\% (0.15, 1.23) for heterozygotes, 0.02\% (0.01, 0.12) for homozygous reference calls, and 0.80\% (0.31, 1.51) for homozygous alternate calls. 
%%


%%%%%%%%%%%%%%%%%%%%%%%%%%%%%%%%%%%%%%%%%%%%%%%%%%%%%%%%%%%%%%%%%%%%%%%%%%%%%%%
\subsection*{3 Population structure}

%
Population structure analyses, AIM, F\raisebox{-.4ex}{\scriptsize ST} and PCA were conducted following methods defined in \cite{Ag1000gConsortium2017}.
%
Doubleton sharing was also conducted following \cite{Ag1000gConsortium2017}, except that due low sample sizes in GHgam (12 samples), GNcol (4) and GQgam (9) were excluded.  
%%


%%%%%%%%%%%%%%%%%%%%%%%%%%%%%%%%%%%%%%%%%%%%%%%%%%%%%%%%%%%%%%%%%%%%%%%%%%%%%%%
\section*{Authors}

%
\textbf{Data analysis group}: Alistair Miles (project lead), Nicholas J. Harding, Giordano Bott\'{a}, Chris S. Clarkson, Tiago Ant\~{a}o, Krzysztof Kozak, Daniel R. Schrider, Andrew D. Kern, Seth Redmond, Igor Sharakhov, Richard D. Pearson, Christina Bergey, Michael C. Fontaine, Martin J. Donnelly, Mara K. N. Lawniczak and Dominic P. Kwiatkowski (chair).
%%

%
\textbf{Partner working group}: Martin J. Donnelly (chair), Diego Ayala, Nora J. Besansky, Austin Burt, Beniamino Caputo, Alessandra della Torre, Michael C. Fontaine, H. Charles J. Godfray, Matthew W. Hahn, Andrew D. Kern, Dominic P. Kwiatkowski, Mara K. N. Lawniczak, Janet Midega, Daniel E. Neafsey, Samantha O'Loughlin, Jo\~{a}o Pinto, Michelle M. Riehle, Igor Sharakhov, Kenneth D. Vernick, David Weetman, Craig S. Wilding and Bradley J. White.
%%

%
\textbf{Sample collections}: Angola: Arlete D. Troco, Jo\~{a}o Pinto; Burkina Faso: Abdoulaye Diabat\'{e}, Samantha O'Loughlin, Austin Burt; Cameroon: Carlo Costantini , Kyanne R. Rohatgi, Nora J. Besansky; Equatorial Guinea: Jorge Cano; Gabon: Nohal Elissa, Jo\~{a}o Pinto; The Gambia: Davis C. Nwakanma, Musa Jawara; Guinea: Boubacar Coulibaly, Michelle M. Riehle, Kenneth D. Vernick; Guinea-Bissau: Jo\~{a}o Pinto, Jo\~{a}o Dinis; Kenya: Janet Midega, Charles Mbogo, Philip Bejon; Mayotte: Gilbert Le Goff, Vincent Robert; Uganda: Craig S. Wilding, David Weetman, Henry D. Mawejje, Martin J. Donnelly; Crosses: David Weetman, Craig S. Wilding, Martin J. Donnelly.
%%

%
\textbf{Sequencing and data production}: Jim Stalker, Kirk Rockett, Eleanor Drury, Daniel Mead, Anna Jeffreys, Christina Hubbart, Kate Rowlands, Alison T. Isaacs, Dushyanth Jyothi, Cinzia Malangone and Maryam Kamali.
%%

%
\textbf{Web application development}: Paul Vauterin, Ben Jeffrey, Ian Wright, Lee Hart and Krzysztof Kluczy\'{n}ski.
%%

%
\textbf{Project coordination}: Victoria Cornelius, Bronwyn MacInnis, Christa Henrichs, Rachel Giacomantonio and Dominic P. Kwiatkowski.



%%%%%%%%%%%%%%%%%%%%%%%%%%%%%%%%%%%%%%%%%%%%%%%%%%%%%%%%%%%%%%%%%%%%%%%%%%%%%%%
%%%%%%%%%%%%%%%%%%%%%%%%%%%%%%%%%%%%%%%%%%%%%%%%%%%%%%%%%%%%%%%%%%%%%%%%%%%%%%%
% TODO enable bibliography
%\printbibliography




\beginsupplement
%%%%%%%%%%%%%%%%%%%%%%%%%%%%%%%%%%%%%%%%%%%%%%%%%%%%%%%%%%%%%%%%%%%%%%%%%%%%%%%
%%%%%%%%%%%%%%%%%%%%%%%%%%%%%%%%%%%%%%%%%%%%%%%%%%%%%%%%%%%%%%%%%%%%%%%%%%%%%%%

\section*{Supplementary tables and figures}

%% Table S1 - crosses.
%
\afterpage{%
%\clearpage
% N.B., for some reason using \newgeometry causes page number to get dropped from the subsequent page, so disable for now - not needed if using \footnotesize.
\newgeometry{margin=2cm}
\thispagestyle{empty}
\begin{table}[h]
  \normalsize
  \centering
  \begin{threeparttable}

  \caption{
%
\textbf{colony crosses}.
}

  \label{table:colony crosses}

  
\begin{tabular}{lllr}
\toprule
Cross ID & 
Mother Colony & 
Father Colony &
N progeny \\
\midrule

18-5 & Ghana & Kisumu/G3 & 20 \\

29-2 & Ghana & Kisumu & 20 \\

36-9 & Ghana & Mali & 20 \\

37-3 & Kisumu & Pimperena & 20 \\

42-4 & Mali & Kisumu/Ghana & 14 \\

45-1 & Mali & Kisumu & 20 \\

46-9 & Pimperena & Mali & 20 \\

47-6 & Mali & Kisumu & 20 \\

73-2 & Akron & Ghana & 19 \\

78-2 & Mali & Kisumu/Ghana & 19 \\

80-2 & Kisumu & Akron & 20 \\

\bottomrule
\end{tabular}



  \end{threeparttable}

\end{table}
\restoregeometry
} % end afterpage
%% end Table 1


\clearpage


\end{document}
