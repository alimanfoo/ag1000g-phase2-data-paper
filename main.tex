\documentclass[a4paper,11pt,abstracton,hidelinks]{scrartcl}

\usepackage[margin=3cm]{geometry}
\usepackage{graphicx}
\usepackage[UKenglish]{babel}
\usepackage{csquotes}
\usepackage[style=numeric,citestyle=numeric,backend=biber,sorting=none,doi=false,url=false]{biblatex}
\usepackage{float}
\usepackage[export]{adjustbox}
\usepackage[T1]{fontenc}
\usepackage{lmodern}
\usepackage[textsize=tiny]{todonotes}
\usepackage[labelsep=period,font=small,labelfont=bf,format=plain]{caption}
\captionsetup[table]{
  position=above,
  belowskip=10pt,
  aboveskip=0pt,
}
\usepackage[group-separator={,}]{siunitx}
\usepackage{booktabs}
\usepackage{pdflscape}
\usepackage{tablefootnote}
\usepackage{authblk}
\usepackage{threeparttable}
\usepackage{afterpage}
\usepackage{lineno}
\linenumbers
\usepackage{setspace}
\usepackage{hyperref}
\doublespacing

\newcommand{\beginsupplement}{%
  \setcounter{table}{0}
  \renewcommand{\thetable}{S\arabic{table}}%
  \setcounter{figure}{0}
  \renewcommand{\thefigure}{S\arabic{figure}}%
}


% TODO create refs.bib
\addbibresource{refs.bib}


\title{
Nucleotide variation in 1,142 genomes of the African malaria vectors \emph{Anopheles gambiae} and \emph{Anopheles coluzzii}
}


\author[1]{\small The \emph{Anopheles gambiae} 1000 Genomes Consortium}
\affil[1]{\footnotesize A list of consortium members appears at the end of the paper}

\begin{document}

\maketitle


%%%%%%%%%%%%%%%%%%%%%%%%%%%%%%%%%%%%%%%%%%%%%%%%%%%%%%%%%%%%%%%%%%%%%%%%%%%%%%%
%%%%%%%%%%%%%%%%%%%%%%%%%%%%%%%%%%%%%%%%%%%%%%%%%%%%%%%%%%%%%%%%%%%%%%%%%%%%%%%
\begin{abstract}

%%
TODO
%%

\end{abstract}


%%%%%%%%%%%%%%%%%%%%%%%%%%%%%%%%%%%%%%%%%%%%%%%%%%%%%%%%%%%%%%%%%%%%%%%%%%%%%%%
%%%%%%%%%%%%%%%%%%%%%%%%%%%%%%%%%%%%%%%%%%%%%%%%%%%%%%%%%%%%%%%%%%%%%%%%%%%%%%%
\section*{Introduction}

%%
TODO
%%


%%%%%%%%%%%%%%%%%%%%%%%%%%%%%%%%%%%%%%%%%%%%%%%%%%%%%%%%%%%%%%%%%%%%%%%%%%%%%%%
%%%%%%%%%%%%%%%%%%%%%%%%%%%%%%%%%%%%%%%%%%%%%%%%%%%%%%%%%%%%%%%%%%%%%%%%%%%%%%%
\section*{Results}

%% 
TODO
%%


%%%%%%%%%%%%%%%%%%%%%%%%%%%%%%%%%%%%%%%%%%%%%%%%%%%%%%%%%%%%%%%%%%%%%%%%%%%%%%%
%%%%%%%%%%%%%%%%%%%%%%%%%%%%%%%%%%%%%%%%%%%%%%%%%%%%%%%%%%%%%%%%%%%%%%%%%%%%%%%
\section*{Discussion}

%%
TODO
%%


%%%%%%%%%%%%%%%%%%%%%%%%%%%%%%%%%%%%%%%%%%%%%%%%%%%%%%%%%%%%%%%%%%%%%%%%%%%%%%%
%%%%%%%%%%%%%%%%%%%%%%%%%%%%%%%%%%%%%%%%%%%%%%%%%%%%%%%%%%%%%%%%%%%%%%%%%%%%%%%
\section*{Methods}


%%%%%%%%%%%%%%%%%%%%%%%%%%%%%%%%%%%%%%%%%%%%%%%%%%%%%%%%%%%%%%%%%%%%%%%%%%%%%%%
\subsection*{1. Population sampling}

%
Ag1000g Phase 2 mosquitoes were collected from natural populations at 33 sites from 23
geographical locations in 13 sub-Saharan African countries. 
%
Samples have been grouped into 16 populations based on collection country and species (@@TODO Figure 1, Table 1).
%
Throughout, we use species nomenclature following Coetzee \textit{et al}. \cite{Coetzee2013};	
%
prior to	 Coetzee	 \textit{et al}., \textit{An. gambiae} was known as \textit{An. gambiae sensu stricto} (S form) and \textit{An. coluzzii} was known as \textit{An. gambiae sensu stricto} (M form).
%
Details of the eighteen collection sites novel to Ag1000g Phase 2 (dates, collection and DNA extraction methods) can be found in the Methods section below \textbf{1.1 Novel Phase 2 natural populations}.
%
Information pertaining to the collection of samples released as part of Ag1000g Phase 1 can be found in the supplementary methods of Ag1000g Consortium (2017) \cite{Ag1000gConsortium2017}.
%%


%%%%%%%%%%%%%%%%%%%%%%%%%%%%%%%%%%%%%%%%%%%%%%%%%%%%%%%%%%%%%%%%%%%%%%%%%%%%%%%
\subsection*{1.1 Novel Phase 2 natural populations}
%
\textbf{C\^{o}te d'Ivoire (CIcol)}: Tiassal\'{e} (-4.82293, 5.89839) is located in the evergreen forest zone of southern C\^{o}te d'Ivoire.
%
The primary agricultural activity is rice cultivation in irrigated fields.
%
High malaria transmission occurs during the rainy seasons, between May and November.
%
Samples were collected as larvae from irrigated rice fields by dipping between May and Septermber 2012.
%
All larvae were reared to adults and females preserved over silica for DNA extraction using Qiagen DNeasy Blood and Tissue Kit (Qiagen Science, MD, USA).
%
Specimens from this site were all \textit{An. coluzzii}
%%

%
\textbf{Equatorial Guinea (GQgam)}: Collections were performed during the rainy season in September, 2002 by overnight CDC light traps in Sacriba of Bioko island (8.7, 3.7).
%
Specimens were stored dry on silica gel and DNA was extracted using the Qiagen DNeasy kit.
%
Specimens contributed from this site were \textit{An. gambiae} females, genotype determined by two assays \cite{Scott1993, Santolamazza2004}.
%
All specimens had the 2L\textsuperscript{+a}/2L\textsuperscript{+a} karyotype as determined by the molecular PCR diagnostics \cite{White2007}. 
%
These mosquitoes represent a population that inhabited Bioko Island before a comprehensive malaria control intervention initiated in February 2004 \cite{Sharp2007}. 
%
After the intervention \textit{An. gambiae} was declining, and more recently almost only \textit{An. coluzzii} can be found \cite{Overgaard2012}.
%%

%
\textbf{France (FRgam)}: Samples were collected as larvae during March-April 2011 in temporary pools by dipping, in Bouyouni (-12.737813, 45.141696), M'Tsamboro Forest Reserve (-12.70271, 45.081091), Combani (-12.778704, 45.142913), Mtsanga Charifou (-12.990662, 45.155673), Karihani Lake forest reserve (-12.796525, 45.121722), and Sada (-12.852147, 45.103891) in Mayotte island.
%
Larvae were stored in 80\% ethanol and genomic DNA was extracted using the Qiagen DNeasy Blood and Tissue Kit (Qiagen Science, MD, USA). 
%
All specimens contributed to Ag1000G phase 2 were \textit{An. gambiae} \cite{Santolamazza2004} with the standard 2L\textsuperscript{+a}/2L\textsuperscript{+a} or inverted 2L\textsuperscript{a}/2L\textsuperscript{a} karyotype as determined by the molecular PCR diagnostics \cite{White2007}.
%
The samples were identified as males or females by the sequencing read coverage of the X chromosome using LookSeq \cite{Manske2009}.
%%

%
\textbf{The Gambia (GM)}: Indoor resting female mosquitoes were collected by pyrethrum spray catch from four hamlets around Njabakunda (-15.9, 13.55), North Bank Region, The Gambia between August and October 2011.
%
The four hamlets were Maria Samba Nyado, Sare Illo Buya, Kerr Birom Kardo, and Kerr Sama Kuma; all are within 1 km of each other.
%
This is an area of unusually high hybridization rates between \textit{An. gambiae s.s.} and \textit{An. coluzzii} \cite{Caputo2008, Nwakanma2013}.
%
Njabakunda village is approximately 30 km to the west of Farafenni town and 4 km away from the Gambia River.
%
The vegetation is a mix of open savannah woodland and farmland.
%
With apparent high gene-flow in the region, it is problematic to assign species to these samples.
%%

%
\textbf{Ghana (GHcol/GHgam)}: Twifo Praso (5.60858, -1.54926) a peri-urban community located in semi-deciduous forest in the Central Region of Ghana.
%
It is an extensive agricultural area characterised by small-scale (vegetable growing) and large-scale commercial farms such as oil palm and cocoa plantations.
%
Mosquito samples were collected as larvae from puddles near farms between September and October, 2012.
%
Madina (5.66849	-0.21928) is suburb of Accra within a coastal savanna zone of Ghana. 
%
It is an urban community characterised by myriad vegetable-growing areas.
%
The vegetation consists of mainly grassland interspersed with dense short thickets often less than 5 m high with a few trees.
%
Specimens were sampled from puddles near roadsides and farms between October and December 2012.
%
Takoradi (4.91217, -1.77397) is the capital city of Western Region of Ghana.
%
It is an urban community located in the coastal savanna zone.
%
Mosquito samples were collected from puddles near road construction and farms between August and September 2012.
%
Koforidua (6.09449, -0.26093) is a capital city of Eastern Region of Ghana and is located in semi-deciduous forest. 
%
It is an urban community characterized by numerous small-scale vegetable farms. 
%
Samples were collected from puddles near road construction and farms between August and September 2012.
%
Larvae from all collection sites were reared to adults and females preserved over silica for DNA extraction using Qiagen DNAEasy kits (Qiagen Science, MD, USA).
%
Both \textit{An. gambiae} and \textit{An. coluzzii} were collected from these sites.
%%

%
\textbf{Guinea-Bissau (GW)}: Mosquitoes were collected in October 2010 using indoor CDC light traps, in the village of Safim (11.956889, -15.649222), ca. 11 km north of Bissau city, the capital of the country.
%
Malaria is hyperendemic in the region and transmitted by members of the Anopheles gambiae complex (Vicente et al., 2017).
%
\textit{Anopheles arabiensis, An. melas, A. coluzzii} and \textit{A. gambiae}, as well as hybrids between the latter two species, are known to occur in the region \cite{Gordicho2014, Vicente2017}.
%
Mosquitoes were preserved individually on 0.5ml micro-tubes filled with silica gel and cotton. DNA extraction was performed by a phenol-chloroform protocol \cite{Donnelly1999}.
%
Guinea-Bissau is another region where defining species is problematic (Vicente), so no species has been assigned here.
%%

%%%%%%%%%%%%%%%%%%%%%%%%%%%%%%%%%%%%%%%%%%%%%%%%%%%%%%%%%%%%%%%%%%%%%%%%%%%%%%%
\subsection*{1.2 Colony populations}

%%%%%%%%%%%%%%%%%%%%%%%%%%%%%%%%%%%%%%%%%%%%%%%%%%%%%%%%%%%%%%%%%%%%%%%%%%%%%%%
%%%%%%%%%%%%%%%%%%%%%%%%%%%%%%%%%%%%%%%%%%%%%%%%%%%%%%%%%%%%%%%%%%%%%%%%%%%%%%%
\section*{Authors}

%
\textbf{Data analysis group}: Alistair Miles (project lead), Nicholas J. Harding, Giordano Bott\'{a}, Chris S. Clarkson, Tiago Ant\~{a}o, Krzysztof Kozak, Daniel R. Schrider, Andrew D. Kern, Seth Redmond, Igor Sharakhov, Richard D. Pearson, Christina Bergey, Michael C. Fontaine, Martin J. Donnelly, Mara K. N. Lawniczak and Dominic P. Kwiatkowski (chair).
%%

%
\textbf{Partner working group}: Martin J. Donnelly (chair), Diego Ayala, Nora J. Besansky, Austin Burt, Beniamino Caputo, Alessandra della Torre, Michael C. Fontaine, H. Charles J. Godfray, Matthew W. Hahn, Andrew D. Kern, Dominic P. Kwiatkowski, Mara K. N. Lawniczak, Janet Midega, Daniel E. Neafsey, Samantha O'Loughlin, Jo\~{a}o Pinto, Michelle M. Riehle, Igor Sharakhov, Kenneth D. Vernick, David Weetman, Craig S. Wilding and Bradley J. White.
%%

%
\textbf{Sample collections}: Angola: Arlete D. Troco, Jo\~{a}o Pinto; Burkina Faso: Abdoulaye Diabat\'{e}, Samantha O'Loughlin, Austin Burt; Cameroon: Carlo Costantini , Kyanne R. Rohatgi, Nora J. Besansky; Equatorial Guinea: Jorge Cano; Gabon: Nohal Elissa, Jo\~{a}o Pinto; The Gambia: Davis C. Nwakanma, Musa Jawara; Guinea: Boubacar Coulibaly, Michelle M. Riehle, Kenneth D. Vernick; Guinea-Bissau: Jo\~{a}o Pinto, Jo\~{a}o Dinis; Kenya: Janet Midega, Charles Mbogo, Philip Bejon; Mayotte: Gilbert Le Goff, Vincent Robert; Uganda: Craig S. Wilding, David Weetman, Henry D. Mawejje, Martin J. Donnelly; Crosses: David Weetman, Craig S. Wilding, Martin J. Donnelly.
%%

%
\textbf{Sequencing and data production}: Jim Stalker, Kirk Rockett, Eleanor Drury, Daniel Mead, Anna Jeffreys, Christina Hubbart, Kate Rowlands, Alison T. Isaacs, Dushyanth Jyothi, Cinzia Malangone and Maryam Kamali.
%%

%
\textbf{Web application development}: Paul Vauterin, Ben Jeffrey, Ian Wright, Lee Hart and Krzysztof Kluczy\'{n}ski.
%%

%
\textbf{Project coordination}: Victoria Cornelius, Bronwyn MacInnis, Christa Henrichs, Rachel Giacomantonio and Dominic P. Kwiatkowski.
%%


%%%%%%%%%%%%%%%%%%%%%%%%%%%%%%%%%%%%%%%%%%%%%%%%%%%%%%%%%%%%%%%%%%%%%%%%%%%%%%%
%%%%%%%%%%%%%%%%%%%%%%%%%%%%%%%%%%%%%%%%%%%%%%%%%%%%%%%%%%%%%%%%%%%%%%%%%%%%%%%
% TODO enable bibliography
%\printbibliography




\beginsupplement
%%%%%%%%%%%%%%%%%%%%%%%%%%%%%%%%%%%%%%%%%%%%%%%%%%%%%%%%%%%%%%%%%%%%%%%%%%%%%%%
%%%%%%%%%%%%%%%%%%%%%%%%%%%%%%%%%%%%%%%%%%%%%%%%%%%%%%%%%%%%%%%%%%%%%%%%%%%%%%%
\section*{Supplementary figures}




\clearpage


\end{document}
