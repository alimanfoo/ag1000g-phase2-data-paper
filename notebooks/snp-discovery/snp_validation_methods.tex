
We used 11 laboratory crosses to estimate the error associated with genotyping from short reads. 
%
We defined errors at sites where given confident parental genotypes and mendelian inheritance, the expected genotypes of progeny are known. 
%
At loci where both parents are homozygous for the reference or alternate allele, we expect the progeny to be homozygous for the reference or alternate allele accordingly.
%
When one parent is homozygous for the reference allele, and the other the alternate allele we expect all progeny to be heterozygous. 
%
Therefore we can generate ascertainment error estimates for heterozygous, homozygous reference and homozygous alternate genotypes.
%
To meet the confidence threshold for homozygous calls, both parents must have 30x coverage, and have no discordant reads.
%
%
Sites are considered erroneously called if at least 10 progeny have genotypes called at that locus, and one or more do not match the expected genotype. 
%
Sites are considered correctly called if at least 10 progeny have genotypes called and all match the expected genotype. 
%
Crosses had between 14-20 progeny, except cross 61-3 (n=8) which was excluded from this analysis.
%
%
Error in ascertainment rates are computed over each cross, and errors are reported as the median over all 11 crosses. 
%
%
Raw error rates were 0.72 (0.39, 2.44) for heterozygotes, 0.07 (0.03, 0.25) for homozygous reference calls, and 0.82 (0.37, 1.47) for homozygous alternate calls. 
Following variant filtering via QC thresholds and the accessibility map these values dropped to 0.26 (0.15, 1.23) for heterozygotes, 0.02 (0.01, 0.12) for homozygous reference calls, and 0.80 (0.31, 1.51) for homozygous alternate calls. 

